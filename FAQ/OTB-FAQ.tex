\documentclass[english]{article}
\usepackage[T1]{fontenc}
\usepackage[latin1]{inputenc}
\usepackage{a4}
\usepackage{amsmath}
\usepackage{babel}
\usepackage{times}
\usepackage{pslatex}
\usepackage[pdftex]{graphicx}
\usepackage[bookmarksnumbered,colorlinks=true,pdftitle={},pdfauthor={Jordi Inglada},pdfsubject={},pdfkeywords={}]{hyperref}
\usepackage{fancyhdr}
\usepackage[normalem]{ulem}
\usepackage{afterpage}
\usepackage{rotating}
%\pagestyle{fancy}
\input epsf



\usepackage{mdwtab}

\usepackage[normal]{subfigure}
\newcommand{\goodgap}{%
	\hspace{\subfigtopskip}%
	\hspace{\subfigbottomskip}}

\def\rit{\hbox{\it I\hskip -2pt R}}
\def\nit{\hbox{\it I\hskip -2pt N}}
\def\cut{\hbox{\it l\hskip -5.5pt C\/}}
\def\zit{\hbox{\it Z\hskip -2pt Z}}
\def\qit{\hbox{\it l\hskip -5.5pt Q}}


\def\BibTeX{{\rm B\kern-.05em{\sc i\kern-.025em b}\kern-.08em
    T\kern-.1667em\lower.7ex\hbox{E}\kern-.125emX}}


\begin{document}


\title{ORFEO Toobox (OTB) Frequently Asked Questions (FAQ).}


\author{OTB Development Team}


\maketitle

\tableofcontents

\section{Introduction}
\subsection{What is OTB?}
OTB, the ORFEO Toolbox is a library of image processing algorithms developed by CNES in the
frame of the ORFEO Accompaniment Program. 
OTB is based on the medical image processing library ITK, \url{http://www.itk.org}, and offers
particular functionnalities for remote sensing image processing in
general and for high spatial resolution images in particular.

OTB provides:
\begin{itemize}
\item image access: optimized read/write access for most of remote sensing
image formats, meta-data access, simple visualization;
\item filtering: blurring, denoising, enhancement;
\item feature extraction: interest points, alignments, lines;
\item image segmentation: region growing, watershed, level sets;
\item classification: K-means, SVM, Markov random fields; 
\item change detection.  
\end{itemize}


Many of these functionnalities are provided by ITK and have been tested
and documented for the use with remote sensing data.

\subsection{What is ORFEO?}
ORFEO stands for Optical ans Radar Federated Earth Observation.  n
2001 a cooperation program was set between France and Italy to develop
ORFEO, an Earth observation dual system with metric resolution: Italy
is in charge of COSMO-Skymed the radar component development, and
France of PLEIADES the optic component.

The PLEIADES optic component is composed of two "small satellites"
(mass of one ton) offering a spatial resolution at nadir of 0.7 m and
a field of view of 20 km. Their great agility enables a daily access
all over the world, essentially for defence and civil security
applications, and a coverage capacity necessary for the cartography
kind of applications at scales better than those accessible to SPOT
family satellites. Moreover, PLEIADES will have stereoscopic
acquisition capacity to meet the fine cartography needs, notably in
urban regions, and to bring more information when used with aerial
photography.

The ORFEO "targeted" acquisition capacities made it a system
particularly adapted to defence or civil security missions, as well as
critical geophysical phenomena survey such as volcanic eruptions,
which require a priority use of the system ressources.


With respect to the constraints of the franco-italian agreement,
cooperations have been set up for the PLEIADES optical component with
Sweden, Belgium, Spain and Austria.

\subsubsection{Where can I get more information about ORFEO?}
At the PLEIADES HR web site: \url{http://smsc.cnes.fr/PLEIADES/}.

\subsection{What is the ORFEO Acompaniment Program?}
Beside the Pleiades (PHR) and Cosmo-Skymed (CSK) systems developments forming ORFEO, the dual and bilateral system (France - Italy) for Earth Observation, the ORFEO Accompaniment Program was set up, to prepare, accompany and promote the use and the exploitation of the images derived from these sensors.

The creation of a preparatory program is needed because of :
\begin{itemize}
  \item  the new capabilities and performances of the ORFEO systems (optical and radar high resolution, access capability, data quality, possibility to acquire simultaneously in optic and radar),
  \item the implied need of new methodological developments : new processing methods, or adaptation of existing methods,
  \item the need to realise those new developments in very close cooperation with the final users, the integration of new products in their systems.
\end{itemize}
  

This program was initiated by CNES mid-2003 and will last until 2009.
It consists in two parts, between which it is necessary to keep a strong interaction:
\begin{itemize}
\item A Methodological part,
\item A Thematic part.
\end{itemize}


This Accompaniment Program uses simulated data (acquired during airborne campaigns) and satellite images quite similar to Pleiades (as QuickBird and Ikonos), used in a communal way on a set of special sites. The validation of specified products and services will be realised with those simulated data

Apart from the initial cooperation with Italy, the ORFEO Accompaniment
Program enlarged to Belgium, with integration of Belgian experts in
the different WG as well as a participation to the methodological
part.

\subsection{Where can I get more information about the ORFEO
  Acompaniment Program?}
Go to the following web site:
\url{http://smsc.cnes.fr/PLEIADES/A_prog_accomp.htm}.

\subsection{Who is responsible for the OTB development?}
The French Centre National d'\'Etudes Spatiales, CNES, initiated the ORFEO
Toolbox and is responsible for the specification of the library. CNES
founds the industrial development contracts and research contracts
needed for the evolution of OTB.

\subsection{Which is the OTB licence?}
OTB is distributed under a free software licence:
\url{http://www.cecill.info/licences/Licence_CeCILL_V2-en.html}.


\section{Getting OTB}
\subsection{Who can download the OTB?}
Anybody can download the OTB at no cost. 
\subsection{Where can I download the OTB?}
Go to \url{http://smsc.cnes.fr/PLEIADES/A_prog_accomp.htm}
 and follow the "ORFEO Toolbox" link. You will have access to the OTB
source code and to the Software User's Guide.
\section{Installing OTB}
\subsection{Which platforms are supported}
OTB is a multi-platform library. It has succesfully been intalled on
the following platforms:
\begin{itemize}
  \item Linux/Unix with GCC (2.95.X, 3.3.X, 4.1.X).
  \item Windows with Visual Studio 7.1
\end{itemize}

Support for the following platforms is planned:
\begin{itemize}
  \item Windows with VC++ 6.
  \item Cygwin.
  \item Windows with Mingw.
\end{itemize}

\subsection{Which libraries/packages are needed before installing
 OTB?}
\begin{itemize}
\item CMake (http://www.cmake.org)
\item GDAL (http://www.gdal.org/)
\item Fltk (http://www.fltk.org)
\end{itemize}
       
\subsection{Main steps}
In order to install OTB on your system follow these steps (in the
given order):
\begin{enumerate}
  \item Install CMake.
  \item Install GDAL.
  \item Install Fltk using the CMake scripts. Do not use the
  \texttt{configure} approach or the project files for Microsoft
  Visual Studio shipped with Fltk.
  \item Install ITK if you do not want to use the ITK version provided
  with OTB. Use CMake for the consiguration.
  \item Install OTB using CMake for the configuration.
\end{enumerate}
\subsubsection{Unix/Linux Platforms}
\begin{enumerate}
\item We assume that you will install everything on a directory called
\texttt{INSTALL\_DIR}, which usually is \texttt{/usr/local}, \texttt{/home/jordi/local} or
whatever you want.
\item Make sure that you have downloaded the source code for:
  \begin{itemize}
  \item CMake (http://www.cmake.org)
  \item GDAL (http://www.gdal.org/)
  \item Fltk (http://www.fltk.org)
  \end{itemize}
    
\item Install GDAL:
  \begin{verbatim}
      cd INSTALL_DIR
      gunzip gdal.1.3.2.tar.gz
      tar xvf gdal.1.3.2.tar
      cd gdal.1.3.2
      ./configure --prefix=INSTALL_DIR
      make
      make install
  \end{verbatim}
      

\item Install CMake:
  \begin{verbatim}
      cd INSTALL_DIR
      gunzip cmake-2.2.3.tar.gz
      tar xvf cmake-2.2.3.tar
      cd cmake-2.2.3
      ./configure --prefix=INSTALL_DIR
      make
      make install
  \end{verbatim}
      In order to properly use cmake, add \texttt{INSTALL\_DIR/bin} to
      your path with \texttt{export PATH=\$PATH:INSTALL\_DIR/bin} or
      something similar.

\item Install Fltk using CMake (do not use the configure script)
  \begin{verbatim}
      cd INSTALL_DIR
      bunzip2 fltk-1.1.7-source.tar.bz2 OR
      gunzip fltk-1.1.7-source.tar.gz
      mkdir Fltk-binary
      cd Fltk-binary
      ccmake ../fltk-1.1.7
      --> follow the CMake instructions, in particular:
          --> set CMAKE_INSTALL_PREFIX to INSTALL_DIR within CMake
	  --> set BUILD_EXAMPLES to ON within CMake
	  --> generate the configuration with 'g'
      make
      make install
      --> check that the examples located in
      INSTALL_DIR/Fltk-binary/bin work, in particular, the fractals
      example which makes use of the OpenGL library needed by OTB.
  \end{verbatim}
      

\item Install OTB
  \begin{verbatim}
      cd INSTALL_DIR
      gunzip OrfeoToolbox-1.0.0.tgz
      tar xvf OrfeoToolbox-1.0.0.tar
      mkdir OTB-Binary
      cd OTB-Binary
      ccmake ../OrfeoToolbox-1.0.0
      --> follow the CMake instructions, in particular:
	  --> set BUILD_EXAMPLES to ON within CMake
	  --> set BUILD_SHARED_LIBS to OFF within CMake
	  --> set BUILD_TESTING to OFF within CMake
	  --> set CMAKE_INSTALL_PREFIX to INSTALL_DIR within CMake
	  --> set GDAL_INCLUDE_DIRS to INSTALL_DIR/include within CMake
	  --> set GDAL_LIBRARY_DIRS to INSTALL_DIR/lib within CMake
	  --> set OTB_USE_EXTERNAL_ITK to OFF within CMake
	  --> set FLTK_DIR to INSTALL_DIR/Fltk-Binary within CMake
	  --> generate the configuration with 'g'
       make
       make install
  \end{verbatim}
      
            
\item That should be all! Otherwise, subscribe to
   otb-users@googlegroups.com and you will get some help.
  
\end{enumerate}

\subsection{Specific platform issues}
\subsubsection{SunOS/HP UX}
Due to a bug in the tar command shipped with some versions of SunOS,
problems may appear when configuring, compiling or installing OTB.

See \url{http://www.gnu.org/software/tar/manual/tar.html#Checksumming} for
details on the bug characterization.

The solution is to use the GNU tar command if it is available on your
system (gtar).



\section{Using OTB}

\section{Getting help}
\subsection{Is there any mailing list?}
Yes. There is a discussion group at
\url{http://groups.google.com/group/otb-users/} where you can get help
on the set up and the use of OTB.

\subsection{Which is the main source of documentation?}
The main source of documentation is the OTB Software Guide which can
be downloaded at \url{http://smsc.cnes.fr/PLEIADES/A_prog_accomp.htm}
following the {\em ORFEO Toolbox} link. It contains
tenths of commented examples which should be a good starting point for
any new OTB user. The code source for these examples is distributed with
the toolbox. Another information source is the on-line API
documentation which is available at the same download area.


\section{OTB's Roadmap}
\subsection{Which will be the next version of OTB?}
\subsection{When will the next version of OTB be available?}
\subsection{What features will the OTB include and when?}
\subsection{When will feature X or Y be included in OTB?}


%\bibliographystyle{plain}

%\bibliography{biblio}


\end{document}















