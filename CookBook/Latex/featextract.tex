\section{Feature extraction}\label{sec:featextract}

As described in the OTB Software Guide, the term {\em Feature Extraction} refers to
techniques aiming at extracting added value information from images. These
extracted items named {\em features} can be local statistical moments, edges,
radiometric indices, morphological and textural properties. For example, such
features can be used as input data for other image processing methods like
{\em Segmentation} and {\em Classification}.

\subsection{Local statistics extraction}\label{ssec:localstatextraction}

This application computes the 4 local statistical moments on every pixel in the
selected channel of the input image, over a specified neighborhood. The output
image is multi band with one statistical moment (feature) per band. Thus, the 4
output features are the Mean, the Variance, the Skewness and the Kurtosis. They are
provided in this exact order in the output image.


The \application{LocalStatisticExtraction} application has the following input parameters:
\begin{itemize}
\item \verb?-in? the input image to compute the features on
\item \verb?-channel? the selected channel index in the input image to be processed (default value is 1)
\item \verb?-radius? the computational window radius (default value is 3 pixels)
\item \verb?-out? the output image containing the local statistical moments
\end{itemize}


The application can be used like this:
\begin{verbatim}
otbcli_LocalStatisticExtraction  -in        InputImage
                                 -channel   1
                                 -radius    3
                                 -out       OutputImage
\end{verbatim}



\subsection{Edge extraction}\label{ssec:edgeextraction}

This application Computes edge features on every pixel in the selected channel
of the input image.

The \application{EdgeExtraction} application has the following input parameters:
\begin{itemize}
\item \verb?-in? the input image to compute the features on
\item \verb?-channel? the selected channel index in the input image to be processed (default value is 1)
\item \verb?-filter? the choice of edge detection method (gradient/sobel/touzi) (default value is gradient)
~~\\
\item \verb?(-filter.touzi.xradius)? the X Radius of the Touzi processing neighborhood (only if filter==touzi) (default value is 1 pixel)
\item \verb?(-filter.touzi.yradius)? the Y Radius of the Touzi processing neighborhood (only if filter==touzi) (default value is 1 pixel)
~~\\
\item \verb?-out? the output mono band image containing the edge features
\end{itemize}


The application can be used like this:
\begin{verbatim}
otbcli_EdgeExtraction  -in        InputImage
                       -channel   1
                       -filter    sobel
                       -out       OutputImage
\end{verbatim}

or like this if filter==touzi:

\begin{verbatim}
otbcli_EdgeExtraction  -in                    InputImage
                       -channel               1
                       -filter                touzi
                       -filter.touzi.xradius  2
                       -filter.touzi.yradius  2 
                       -out                   OutputImage
\end{verbatim}




\subsection{Radiometric indices extraction}\label{ssec:Radiomindextraction}

This application computes radiometric indices using the channels of the input
image. The output is a multi band image into which each channel is one of
the selected indices.


The \application{RadiometricIndices} application has the following input parameters:
\begin{itemize}
\item \verb?-in? the input image to compute the features on
\item \verb?-out? the output image containing the radiometric indices
\item \verb?-channels.blue? the Blue channel index in the input image (default value is 1)
\item \verb?-channels.green? the Green channel index in the input image (default value is 1)
\item \verb?-channels.red? the Red channel index in the input image (default value is 1)
\item \verb?-channels.nir? the Near Infrared channel index in the input image (default value is 1)
\item \verb?-channels.mir? the Mid-Infrared channel index in the input image (default value is 1)
\item \verb?-list? the list of available radiometric indices (default value is Vegetation:NDVI)
\end{itemize}

The available radiometric indices to be listed into -list with their relevant
channels in brackets are:

\begin{verbatim}
Vegetation:NDVI - Normalized difference vegetation index (Red, NIR)
Vegetation:TNDVI - Transformed normalized difference vegetation index (Red, NIR)
Vegetation:RVI - Ratio vegetation index (Red, NIR)
Vegetation:SAVI - Soil adjusted vegetation index (Red, NIR)
Vegetation:TSAVI - Transformed soil adjusted vegetation index (Red, NIR)
Vegetation:MSAVI - Modified soil adjusted vegetation index (Red, NIR)
Vegetation:MSAVI2 - Modified soil adjusted vegetation index 2 (Red, NIR)
Vegetation:GEMI - Global environment monitoring index (Red, NIR)
Vegetation:IPVI - Infrared percentage vegetation index (Red, NIR)

Water:NDWI - Normalized difference water index (Gao 1996) (NIR, MIR)
Water:NDWI2 - Normalized difference water index (Mc Feeters 1996) (Green, NIR)
Water:MNDWI - Modified normalized difference water index (Xu 2006) (Green, MIR)
Water:NDPI - Normalized difference pond index (Lacaux et al.) (MIR, Green)
Water:NDTI - Normalized difference turbidity index (Lacaux et al.) (Red, Green)

Soil:RI - Redness index (Red, Green)
Soil:CI - Color index (Red, Green)
Soil:BI - Brightness index (Red, Green)
Soil:BI2 - Brightness index 2 (NIR, Red, Green)
\end{verbatim}

The application can be used like this, which leads to an output image with 3 bands,
respectively with the Vegetation:NDVI, Vegetation:RVI and Vegetation:IPVI
radiometric indices in this exact order:

\begin{verbatim}
otbcli_RadiometricIndices -in             InputImage
                          -out            OutputImage
                          -channels.red   3
                          -channels.green 2
                          -channels.nir   4
                          -list           Vegetation:NDVI Vegetation:RVI
                                          Vegetation:IPVI 
\end{verbatim}


or like this, which leads to a single band output image with the Water:NDWI2 radiometric indice:
\begin{verbatim}
otbcli_RadiometricIndices -in             InputImage
                          -out            OutputImage
                          -channels.red   3
                          -channels.green 2
                          -channels.nir   4
                          -list           Water:NDWI2 
\end{verbatim}

        




\subsection{Morphological features extraction}\label{ssec:morphofeatextraction}

Morphological features can be highlighted by using image filters based on
mathematical morphology either on binary or gray scale images.

\subsubsection{Binary morphological operations}

This application performs binary morphological operations (dilation, erosion,
opening and closing) on a mono band image with a specific structuring element (a
ball or a cross) having one radius along X and another one along Y. NB: the cross
shaped structuring element has a fixed radius equal to 1 pixel in both X and Y
directions.

The \application{BinaryMorphologicalOperation} application has the following input parameters:
\begin{itemize}
\item \verb?-in? the input image to be filtered
\item \verb?-channel? the selected channel index in the input image to be processed (default value is 1)
\item \verb?-structype? the choice of the structuring element type (ball/cross) (default value is ball)
\item \verb?(-structype.ball.xradius)? the ball structuring element X Radius (only if structype==ball) (default value is 5 pixels)
\item \verb?(-structype.ball.yradius)? the ball structuring element Y Radius (only if structype==ball) (default value is 5 pixels)
\item \verb?-filter? the choice of the morphological operation (dilate/erode/opening/closing) (default value is dilate)
\item \verb?(-filter.dilate.foreval)? the foreground value for the dilation (idem for filter.erode/opening/closing) (default value is 1)
\item \verb?(-filter.dilate.backval)? the background value for the dilation (idem for filter.erode/opening/closing) (default value is 0)
\item \verb?-out? the output filtered image
\end{itemize}


The application can be used like this:
\begin{verbatim}
otbcli_BinaryMorphologicalOperation  -in                     InputImage
                                     -channel                1
                                     -structype              ball
                                     -structype.ball.xradius 10
                                     -structype.ball.yradius 5
                                     -filter                 opening
                                     -filter.opening.foreval 1.0
                                     -filter.opening.backval 0.0
                                     -out                    OutputImage
\end{verbatim}





\subsubsection{Gray scale morphological operations}

This application performs morphological operations (dilation, erosion,
opening and closing) on a gray scale mono band image with a specific structuring
element (a ball or a cross) having one radius along X and another one along Y. NB:
the cross shaped structuring element has a fixed radius equal to 1 pixel in both X
and Y directions.

The \application{GrayScaleMorphologicalOperation} application has the following input parameters:
\begin{itemize}
\item \verb?-in? the input image to be filtered
\item \verb?-channel? the selected channel index in the input image to be processed (default value is 1)
\item \verb?-structype? the choice of the structuring element type (ball/cross) (default value is ball)
\item \verb?(-structype.ball.xradius)? the ball structuring element X Radius (only if structype==ball) (default value is 5 pixels)
\item \verb?(-structype.ball.yradius)? the ball structuring element Y Radius (only if structype==ball) (default value is 5 pixels)
\item \verb?-filter? the choice of the morphological operation (dilate/erode/opening/closing) (default value is dilate)
\item \verb?-out? the output filtered image
\end{itemize}


The application can be used like this:
\begin{verbatim}
otbcli_GrayScaleMorphologicalOperation  -in                     InputImage
                                        -channel                1
                                        -structype              ball
                                        -structype.ball.xradius 10
                                        -structype.ball.yradius 5
                                        -filter                 opening
                                        -out                    OutputImage
\end{verbatim}



\subsection{Textural features extraction}\label{ssec:texturefeatextraction}

Texture features can be extracted with the help of image filters based on
texture analysis methods like Haralick and structural feature set (SFS).

\subsubsection{Haralick texture features}

This application computes Haralick, advanced and higher order texture features on
every pixel in the selected channel of the input image. The output image is multi
band with a feature per band.

The \application{HaralickTextureExtraction} application has the following input parameters:
\begin{itemize}
\item \verb?-in? the input image to compute the features on
\item \verb?-channel? the selected channel index in the input image to be processed (default value is 1)
\item \verb?-texture? the texture set selection [simple/advanced/higher] (default value is simple)
\item \verb?-parameters.min? the input image minimum (default value is 0) 
\item \verb?-parameters.max? the input image maximum (default value is 255)
\item \verb?-parameters.xrad? the X Radius of the processing neighborhood (default value is 2 pixels)
\item \verb?-parameters.yrad? the Y Radius of the processing neighborhood (default value is 2 pixels)
\item \verb?-parameters.xoff? the $\Delta$X Offset for the co-occurrence computation (default value is 1 pixel)
\item \verb?-parameters.yoff? the $\Delta$Y Offset for the co-occurrence computation (default value is 1 pixel)
\item \verb?-parameters.nbbin? the number of bin per axis for histogram generation (default value is 8)
\item \verb?-out? the output multi band image containing the selected texture features (one feature per band)
\end{itemize}


The available values for -texture with their relevant features are:

\begin{itemize}
\item \verb?-texture=simple:?
In this case, 8 local Haralick textures features will be processed.
The 8 output image channels are: Energy, Entropy, Correlation, Inverse Difference
Moment, Inertia, Cluster Shade, Cluster Prominence and Haralick Correlation. They
are provided in this exact order in the output image. Thus, this application
computes the following Haralick textures over a sliding windows with user defined
radius: (where $ g(i, j) $ is the element in cell i, j of a normalized Gray Level
Co-occurrence Matrix (GLCM)):

"Energy" $ = f_1 = \sum_{i, j}g(i, j)^2 $

"Entropy" $ = f_2 = -\sum_{i, j}g(i, j) \log_2 g(i, j)$, or 0 if $g(i, j) = 0$

"Correlation" $ = f_3 = \sum_{i, j}\frac{(i - \mu)(j - \mu)g(i, j)}{\sigma^2} $

"Inverse Difference Moment" $= f_4 = \sum_{i, j}\frac{1}{1 + (i - j)^2}g(i, j) $

"Inertia" $ = f_5 = \sum_{i, j}(i - j)^2g(i, j) $ (sometimes called "contrast")

"Cluster Shade" $ = f_6 = \sum_{i, j}((i - \mu) + (j - \mu))^3 g(i, j) $

"Cluster Prominence" $ = f_7 = \sum_{i, j}((i - \mu) + (j - \mu))^4 g(i, j) $

"Haralick's Correlation" $ = f_8 = \frac{\sum_{i, j}(i, j) g(i, j) -\mu_t^2}{\sigma_t^2} $
where $\mu_t$ and $\sigma_t$ are the mean and standard deviation of the row (or
column, due to symmetry) sums.

Above, $ \mu = $ (weighted pixel average) $ = \sum_{i, j}i \cdot g(i, j) = \sum_{i, j}j \cdot g(i, j) $
(due to matrix symmetry), and $ \sigma = $ (weighted pixel variance) $ = \sum_{i, j}(i - \mu)^2 \cdot g(i, j) = \sum_{i, j}(j - \mu)^2 \cdot g(i, j) $
(due to matrix symmetry).

\item \verb?-texture=advanced:?
In this case, 9 advanced texture features will be processed. The 9 output image
channels are: Mean, Variance, Sum Average, Sum Variance, Sum Entropy, Difference of
Entropies, Difference of Variances, IC1 and IC2. They are provided in this exact
order in the output image.

\item \verb?-texture=higher:?
In this case, 11 local higher order statistics texture coefficients based on the
grey level run-length matrix will be processed. The 11 output image channels are:
Short Run Emphasis, Long Run Emphasis, Grey-Level Nonuniformity, Run Length
Nonuniformity, Run Percentage, Low Grey-Level Run Emphasis, High Grey-Level Run
Emphasis, Short Run Low Grey-Level Emphasis, Short Run High Grey-Level Emphasis,
Long Run Low Grey-Level Emphasis and Long Run High Grey-Level Emphasis. They are
provided in this exact order in the output image. Thus, this application computes
the following Haralick textures over a sliding window with user defined radius:
(where $ p(i, j) $ is the element in cell i, j of a normalized Run Length Matrix,
$n_r$ is the total number of runs and $n_p$ is the total number of pixels):

"Short Run Emphasis" $ = SRE = \frac{1}{n_r} \sum_{i, j}\frac{p(i, j)}{j^2} $

"Long Run Emphasis" $ = LRE = \frac{1}{n_r} \sum_{i, j}p(i, j) * j^2 $

"Grey-Level Nonuniformity" $ = GLN = \frac{1}{n_r} \sum_{i} \left( \sum_{j}{p(i, j)} \right)^2 $

"Run Length Nonuniformity" $ = RLN = \frac{1}{n_r} \sum_{j} \left( \sum_{i}{p(i, j)} \right)^2 $

"Run Percentage" $ = RP = \frac{n_r}{n_p} $

"Low Grey-Level Run Emphasis" $ = LGRE = \frac{1}{n_r} \sum_{i, j}\frac{p(i, j)}{i^2} $

"High Grey-Level Run Emphasis" $ = HGRE = \frac{1}{n_r} \sum_{i, j}p(i, j) * i^2 $

"Short Run Low Grey-Level Emphasis" $ = SRLGE = \frac{1}{n_r} \sum_{i, j}\frac{p(i, j)}{i^2 j^2} $

"Short Run High Grey-Level Emphasis" $ = SRHGE = \frac{1}{n_r} \sum_{i, j}\frac{p(i, j) * i^2}{j^2} $

"Long Run Low Grey-Level Emphasis" $ = LRLGE = \frac{1}{n_r} \sum_{i, j}\frac{p(i, j) * j^2}{i^2} $

"Long Run High Grey-Level Emphasis" $ = LRHGE = \frac{1}{n_r} \sum_{i, j} p(i, j) i^2 j^2 $

\end{itemize}




The application can be used like this:
\begin{verbatim}
otbcli_HaralickTextureExtraction  -in             InputImage
                                  -channel        1
                                  -texture        simple
                                  -parameters.min 0
                                  -parameters.max 255
                                  -out            OutputImage
\end{verbatim}





\subsubsection{SFS texture extraction}

This application computes Structural Feature Set textures on every pixel in the
selected channel of the input image. The output image is multi band with a feature
per band. The 6 output texture features are SFS'Length, SFS'Width, SFS'PSI,
SFS'W-Mean, SFS'Ratio and SFS'SD. They are provided in this exact order in the
output image.

It is based on line direction estimation and described in the following
publication. Please refer to Xin Huang, Liangpei Zhang and Pingxiang Li
publication, Classification and Extraction of Spatial Features in Urban Areas Using
High-Resolution Multispectral Imagery. IEEE Geoscience and Remote Sensing Letters,
vol. 4, n. 2, 2007, pp 260-264.

The texture is computed for each pixel using its neighborhood. User can set the
spatial threshold that is the max line length, the spectral threshold that is the
max difference authorized between a pixel of the line and the center pixel of the
current neighborhood. The adjustement constant alpha and the ratio Maximum
Consideration Number, which describes the shape contour around the central pixel,
are used to compute the $w - mean$ value.


The \application{SFSTextureExtraction} application has the following input parameters:
\begin{itemize}
\item \verb?-in? the input image to compute the features on
\item \verb?-channel? the selected channel index in the input image to be processed (default value is 1)
\item \verb?-parameters.spethre? the spectral threshold (default value is 50) 
\item \verb?-parameters.spathre? the spatial threshold (default value is 100 pixels)
\item \verb?-parameters.nbdir? the number of directions (default value is 20)
\item \verb?-parameters.alpha? the alpha value (default value is 1)
\item \verb?-parameters.maxcons? the ratio Maximum Consideration Number (default value is 5)
\item \verb?-out? the output multi band image containing the selected texture features (one feature per band)
\end{itemize}
 

The application can be used like this:
\begin{verbatim}
otbcli_SFSTextureExtraction -in             InputImage
                            -channel        1
                            -out            OutputImage
\end{verbatim}
