\section{Optical pre-processing}\label{sec:optpreproc}

This section present various pre-processing tasks that can be done
using \app or \mont.

\subsection{Optical radiometric calibration}\label{ssec:optcal}

In remote sensing imagery, pixel values are called DN (for Digital
Numbers) and can not be physically interpreted and compared: they are
influenced by various factors such as the amount of light flowing
trough the sensor, the gain of the detectors and the analogic to
numeric converter.

Depending on the season, the light and atmospheric conditions, to
position of the sun or the sensor internal parameters, these DN can
drastically change for a given pixel (apart from any ground change
effects). Moreover, these effect are not uniform over the spectrum:
for instance aerosol amount and type has usually more impact on the
blue channel.

Therefore, it is necessary to calibrate the pixel values before any
physical interpretation is made out of them. In particular, this
processing is mandatory before any comparison of pixel spectrum
between several images (from the same sensor), and to train a
classifier without dependence to the atmospheric conditions at the
acquisition time.

Calibrated values are called surface reflectivity, which is a ratio
denoting the fraction of light that is reflected by the underlying
surface in the given spectral range. As such, its values lie in the
range $[0,1]$. For convenience, images are often stored in thousandth
of reflectivity, so that they can be encoded with an integer type.
Two levels of calibration are usually distinguished:

\begin{itemize}
\item The first level is called \emph{Top Of Atmosphere (TOA)}
  reflectivity. It takes into account the sensor gain, sensor spectral
  response and the solar illumination.
\item The second level is called \emph{Top Of Canopy (TOC)}
  reflectivity. In addition to sensor gain and solar illumination, it
  takes into account the optical thickness of the atmosphere, the
  atmospheric pressure, the water vapor amount, the ozone amount, as
  well as the composition and amount of aerosol gasses.
\end{itemize}

This transformation can be done either with \app or with
\mont. Sensor-related parameters such as gain, date, spectral
sensitivity and sensor position are seamlessly read from the image
metadata. Atmospheric parameters can be tuned by the user. Supported
sensors are :
\begin{itemize}
\item SPOT5,
\item QuickBird,
\item Ikonos,
\item WorldView1,
\item WorldView2,
\item Formosat.
\end{itemize}

\subsubsection{Optical calibration with \app}

The \application{otbOpticalCalibration-cli} application from \app
allows to perform command-line optical calibration. The mandatory
parameters are the input and output images and the level of
calibration (either TOA or TOC). All other parameters are
optional. The output images are expressed in thousandth of
reflectivity using a 16 bits unsigned integer type.

A basic TOA calibration task can be performed with the following command :

\begin{verbatim}
otbOpticalCalibration-cli -in  input_image -out output_image -level TOA
\end{verbatim}

A basic TOC calibration task can be performed with the following command :

\begin{verbatim}
otbOpticalCalibration-cli -in  input_image -out output_image -level TOC
\end{verbatim}

\subsubsection{Optical calibration with \mont}


\subsection{Pan-sharpening}\label{ssec:pxs}

todo.

\subsection{Digital Elevation Model management}\label{ssec:dem}

todo.

\subsection{Ortho-rectification and map projections}\label{ssec:ortho}

todo.

\subsection{Residual registration}\label{ssec:registration}

todo.
