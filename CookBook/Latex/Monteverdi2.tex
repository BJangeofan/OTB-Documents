\chapter{A brief tour of Monteverdi2}\label{chap:Monteverdi2} 

\section{Introduction}\label{sec:montintro}
TODO 

\section{Installation}\label{sec:montinstall} 
  
Installation of \montNew is very simple. Standard installer packages are available on the main platforms thanks to OTB-Developpers and external users. These packages are available few days after the release. Get the latest information on binary packages on the \website in the section download.

We will discribe in the following sections the way to install \montNew on:
\begin{itemize}
\item Windows platform (XP/Seven)
\item Ubuntu 12.04 and higher
\item MacOSX 10.8
\end{itemize}

If you want build from source or if we don't provide packages for your system, some informations are available into the \sg, in the section \textbf(Building from Source)

\subsection{Windows XP/Seven/8.1}
For Windows XP/Seven/8.1 users, there is a classical standalone installation program for \montNew, available from the \download after each release of \montNew. 

\subsection{MacOS X}
A standard DMG package is available for \montNew for MacOS X 10.8. Please go the \download.
Click on the file to launch \montNew. We will provide in the next release a package for MacOSX 10.9.

\subsection{Ubuntu 12.04 and higher}
For Ubuntu 12.04 and higher, \montNew package may be available as Debian package through APT repositories.

Since release 0.2, \montNew packages are available in the
\href{https://launchpad.net/~ubuntugis/+archive/ubuntugis-unstable}{ubuntugis-unstable} repository.

You can add it by using these command-lines:
\begin{verbatim}
sudo aptitude install add-apt-repository
sudo apt-add-repository ppa:ubuntugis/ubuntugis-unstable
\end{verbatim}

Now run:
\begin{verbatim}
sudo aptitude install monteverdi2
\end{verbatim}

If you are using \emph{Synaptic}, you can add the repository, update and install the package through the
graphical interface.

\textbf{apt-add-repository} will try to retrieve the GPG keys of the
repositories to certify the origin of the packages. If you are behind a http
proxy, this step won't work and apt-add-repository will stall and eventually
quit. You can temporarily ignore this error and proceed with the update
step. Following this, aptitude update will issue a warning about a signature
problem. This warning won't prevent you from installing the packages.


\subsection{What does it look like?}
TODO

