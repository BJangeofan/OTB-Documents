\chapter{Using OTB applications}\label{chap:WrappedApplications}

\section{Introduction}\label{sec:wrappedAppliIntro}
The \app package (see \ref{chap:otb-applications}) has brought a nice set of 
applications, along with several tools to launch them (command line, Qt,...).
As for the future of these applications, it has been decided to integrate 
them inside the \otb library. This migration has been an opportunity to 
enhance all the framework around the applications: not only new features have
 been added, but the interface with the developer has also been simplified. 
The new framework has inherited the wrappers from \app, and new ones have been
added. The development philosophy behind these applications is to provide users 
with modular functional blocs for remote sensing, ready to be integrated in any
environment. 

Because the applications are now a part of the library, their installation doesn't
require much effort: when building the \otb library, you can activate the applications 
with the CMake boolean option \code{BUILD\_APPLICATIONS}.

\section{List of applications}\label{sec:wrappedAppliList}
The documentation of the available applications is accessible 
\href{http://orfeo-toolbox.org/Applications}{here}. Most of the old applications have
been migrated. They are sorted by categories.

\section{Available wrappers}\label{sec:wrappedAppliWrappers}

\subsection{Command line}\label{sec:wrappedAppliCmdLine}
By default, the applications can be called with the command line launcher. This launcher
is built in your \code{OTB\_DIR/bin} directory. It needs at least two arguments: the 
application name and the path to the \code{OTB\_DIR/bin} directory. Any additional argument
will be given to the application itself. You can ask the application to print its help
message:
\begin{verbatim}
otbApplicationLauncherCommandLine Rescale OTB_DIR/bin -help
\end{verbatim}
The help message (but also the \href{http://orfeo-toolbox.org/Applications}{documentation})
will give you the list of available parameters. Each parameter must be set by giving the 
corresponding key followed by its value:
\begin{verbatim}
otbApplicationLauncherCommandLine Rescale OTB_DIR/bin -in QB_Toulouse_Ortho_PAN.tif -out QB_Toulouse_rescaled.tif -outmin 0 -outmax 255
\end{verbatim}

If the application has sub-parameters (i.e. parameters contained in other parameters), 
their key must be prefixed by their full tree path. For instance, the application 
\code{OrthoRectification} has a paramater for bicubic interpolation radius whose key is \code{radius}.
This parameter should be called with the path \code{-interpolator.bco.radius}.

Note that some types of parameters allow you to give several values after the key (they must
be separated with whitespaces).

\subsection{Other wrappers}\label{sec:wrappedAppliOtherWrap}
If you want to use an other wrapper, you have to activate the corresponding CMake 
options when building the library:
\begin{itemize}
  \item Enable \code{WRAP\_QT} to build a Qt application launcher. It opens a GUI which 
  allows you to set the parameters and execute the application. This launcher only needs 
  the same two arguments as the command line:
  \begin{verbatim}
  otbApplicationLauncherQt Rescale OTB_DIR/bin
  \end{verbatim}
  It displays a window with several tabs. \textbf{[Parameters]} is where you set the parameters and execute the application.
  \textbf{[Logs]} is where you see the informations given by the application during its execution. \textbf{[Progress]} is 
  where you see a progress bar of the execution (not available for all applications). \textbf{[Documentation]} is where you
  find a summary of the application documentation.
   
  \item \code{WRAP\_PYTHON} : \textbf{TODO}
  \item \code{WRAP\_PYQT} : \textbf{TODO}
  \item \code{WRAP\_JAVA} : \textbf{TODO}
\end{itemize}

