\chapter{A brief tour of OTB-Applications}\label{chap:otb-applications}

\section{Introduction}\label{sec:appintro}

OTB-Applications is perhaps the older package of the \otb
suite after the OTB package itself. Since the \otb is a
library providing remote sensing functionalities, the only
applications that were distributed at the beginning were the examples
from the Software Guide and the tests. These applications are very
useful for the developer because their code is very short and only
demonstrates one functionality at a time, but in many cases a real
application would require combining together two or more functions
from the \otb, and providing a higher level interface to
handle parameters, input and output data and communication with the
user nicely.

The \app package was originally designed to provide applications
performing simple remote sensing tasks, more complex than simple
examples from the Software Guide, and with a more user-friendly
interface (either graphical or command-line), to demonstrate the use
of the \otb functions. The most popular applications are maybe the
\application{otbImageViewerManager}, which allows to open a collection
of images and navigate in them, and the
\application{otbSupervisedClassificationApplication}, which allows to
delineate training regions of interest on the image and classify the
image with a SVM classifier trained with these regions. During the
first 3 years of the \otb development, many more applications have
been added to this package, to perform various tasks. Most of them
come with a graphical user interface, apart from some small utilities
that are command-line.  For a complete list of these applications,
please refer to section~\ref{sec:appstruct}.

The development and release of the \mont software (see
chapter~\ref{chap:monteverdi} at the end of year 2009 changed a lot of
things for the \app package: most of non-developer users were looking
for quite a long time for an applications providing \otb
functionalities under a unified graphical interface. Many applications
from the \app package were integrated to \mont as modules, and the
\app package lost a lot of its usefulness. No more applications were
added to the package and it was barely maintained, as new graphical
tools were directly embedded within \mont.

Then, some people started to regain interest in the \app package. \mon
is a great tool to perform numerous remote sensing and image
processing task in a minute, but it is not well adapted to heavier
(and longer) processing, scripting and batch processing. Therefore, in
2010 the \app package has been revamped: old applications have been
moved to a legacy folder for backward compatibility, and the
development team started to populate the package with compact
command-line tools to perform various heavy processing tasks. The
package is now rich of more than 40 tools, though not very well known
from the users for now. Although for now only the commmand-line
interface is fully functional, a work in progress aims at wrapping these
command-line tools to also offer QT graphical interfaces and integration
with the \qgis software as well as with other environment such as python,
IDL or Matlab (and with \mont).

\section{Installation}\label{sec:appinstall}


\section{Structure of the package}\label{sec:appstruct}
