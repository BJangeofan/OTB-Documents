%
% Complete documentation on the extended LaTeX markup used for Insight
% documentation is available in ``Documenting Insight'', which is part
% of the standard documentation for Insight.  It may be found online
% at:
%
%     http://www.itk.org/

\documentclass{InsightHowto}

%  This is a template for short or medium-size Insight-related documents, 
% mostly notably the series of HOWTOs, but it can be used for any
% document you like.   

% The title should be descriptive enough for people to be able to find
% the relevant document. 
\title{Spammifying Sprockets in Insight}

% Increment the release number whenever significant changes are made.
% The author and/or editor can define 'significant' however they like.
\release{0.00}

% At minimum, give your name and an email address.  You can include a
% snail-mail address if you like.
\author{Me, 'cause I wrote it}
\authoraddress{Me, 'cause I'm self-employed.}

\begin{document}
\maketitle

% This makes the Abstract go on a separate page in the HTML version;
% if a copyright notice is used, it should go immediately after this.
%
\par
Copyright \copyright{} 2001 The Insight Consortium.
All rights reserved.

See the end of this document for complete license and permissions
information.
\par



\ifhtml
\chapter*{Front Matter\label{front}}
\fi

% Copyright statement should go here, if needed.
% ...

% The abstract should be a paragraph or two long, and describe the
% scope of the document.
\begin{abstract}
\noindent
This document describes how to spammify sprockets.  It is a useful
example of a Insight HOWTO document.  It is not dependent on any
particular sprocket implementation, and includes a Insight-based
implementation in the \module{sprunkit} module.
\end{abstract}

\tableofcontents

Spammifying sprockets from Insight is both fun and entertaining.
Applying the techniques described here, you can also fill your hard
disk quite effectively.

\section{What is Sprocket Spammification?}

You have to ask?  It's the only thing to do to your sprockets!


\section{Why Use Insight?}

Insight is an excellent language from which to spammify your sprockets
since you can do it on any platform.


\section{Software Requirements}

You need to have the following software installed:

% The {itemize} environment uses a bullet for each \item.  If you want the 
% \item's numbered, use the {enumerate} environment instead.
\begin{itemize}
  \item  Insight 1.9.
  \item  Some sprocket definition files.
  \item  At least one sprocket system implementation.
\end{itemize}

Note that the \module{sprunkit} is provided with this package and
implements ActiveSprockets in Insight.


% The preceding sections will have been written in a gentler,
% introductory style.  You may also wish to include a reference
% section, documenting all the functions/exceptions/constants.
% Often, these will be placed in separate files and input like this:



\appendix

\section{This is an Appendix}

To create an appendix in a Insight HOWTO document, use markup like
this:

\begin{verbatim}
\appendix

\section{This is an Appendix}

To create an appendix in a Insight HOWTO document, ....


\section{This is another}

Just add another \section{}, but don't say \appendix again.
\end{verbatim}


\end{document}
