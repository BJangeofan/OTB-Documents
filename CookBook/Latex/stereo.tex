\subsection{Stereo reconstruction from VHR optical image pair}\label{sec:stereoreconstruction}

This section describes how to convert pair of images into elevation information.

\subsubsection{First step: estimate epipolar geometry transformation}\label{ssec:epipolar}
The aim of this application is to generate resampling grids to transform images
in epipolar geometry.  Epipolar geometry is the geometry of stereo vision (see
\href{http://en.wikipedia.org/wiki/Epipolar_geometry}). The operation of stereo
rectification determines a transformation of each image plane such that pairs of
conjugate epipolar lines become collinear and parallel to one of the image axes.

After applying this transformation, it reduces the problem of elevation (or
stereo correspondences determination) to a 1-D problem.  We've got two images
image1 and image2 over the same area (the stereo pair) and we assume that we
know the localization functions (forward and inverse) associated for each of
these images.

The forward function allows to go from the image referential to the geographic
referential:
\begin{equation}
  (long,lat) = f^{forward}_{image1}(i,j,h)
\end{equation}

where h is the elevation hypothesis, (i, j) are the pixel coordinates in image1
and (long,lat) are geographic coordinates.  As you can imagine, the inverse
function allows to go from geographic coordinates to the image geometry.

For the second image:

\begin{equation}
   (long,lat,h) = f^{inverse}_{image2}(i,j)
\end{equation}

Using jointly the forward and inverse functions from the image pair, we can
construct a co-localisation function $f_{image1 \rightarrow image2}$ between a
the position of a pixel in the first and its position in the second one:

\begin{equation}
(i_{image2},j_{image2}) = f_{image1 \rightarrow image2} (i_{image1} , j_{image1} , h)
\end{equation}

The expression of this function is:

\begin{equation}
f_{image1 \rightarrow image2} (i_{image1} , j_{image1} , h) =  f^{inverse}_{image2} f^{forward}_{image1}((i_{image1} , j_{image1}), h)
\end{equation}

The expression is not really important, what we need to understand is that if we
are now able to determine for a given pixel in image1 the corresponding pixel in
image2.  As we know the expression of co-localisation function between images,
we've got the information about the elevation (variable h in the equation)!

We've got now the mathematical basis to understand how 3-D information can be
extracted by examination of the relative positions of objects in the two 2-D
images.

The construction of the two epipolar grids is a little bit more complicated in
case of VHR optical images.That's because most of passive remote sensing from
space use a push broom sensor, which corresponds to a line of sensors arranged
perpendicular to the flight direction of the spacecraft. This acquisition
configuration implies a slightly different strategy for stereo-rectification
(see details here :
\href{http://en.wikipedia.org/wiki/Epipolar_geometry#Epipolar_geometry_of_pushbroom_sensor}).

Let's examine now how to use the \application{StereoRectificationGridGenerator}
application to produce two images which are \textgreater{deformation grids} for
image1 and image2.

\begin{verbatim}
otbcli_StereoRectificationGridGenerator -io.inleft image1.tif
                                        -io.inright image1.tif
                                        -epi.elevation.avg.value 100
                                        -epi.step 10
                                        -io.outimage1 outimage1_grid.tif
                                        -io.outright outimage1_grid.tif
\end{verbatim}

The application estimates a displacement to apply for each pixel in the two
input images to obtain epipolar geometry.You can see that the application can
accept a `step' parameter to estimate displacements on a coarser grid. Here we
estimate the displacements every 10 pixel. That's because in most cases with a
pair of VHR and a small angle between the two images, this grid is almost
regular.  Moreover, the implementation is not 'streamable' and use potentially a
lot of memory. So that, it is generally a good idea to estimate the displacement
grid at a coarser resolution.

The application outputs the size of the output image in epipolar
geometry. \textbf{Note these values}, we will use it at the next step to
resample the two images in epipolar geometry.

In my case, I've got:

\begin{verbatim}
Output parameters value:
epi.rectsizex: 4464
epi.rectsizey: 2958
epi.baseline: 4.772260666
\end{verbatim}

Moreover the epi.baseline parameter provides the mean value, in meters, of the
baseline to sensor altitude ratio. It can be used to convert disparities to
physical elevation, since a disparity of one pixel will correspond to an
elevation offset of this value with respect to the mean elevation.

Let's move forward to the resampling in epipolar geometry.

\subsubsection{Resample images in epipolar geometry}

The prior application generates two grids of displacements. The
\application{GridBasedImageResampling} allows to resample the two input images
in the epipolar geometry using these grids.  These grids are intermediary result
not really useful as it in most cases. This second step ``only'' consists in
applying the transformation and resample both images but this application can be
useful in a lot of other cases.

The two commands to generate epipolar images are:
\begin{verbatim}
otbcli_GridBasedImageResampling -io.in image1.tif
                                -io.out image1_epipolar.tif
                                -grid.in outimage1_grid.tif
                                -out.sizex 4464
                                -out.sizey 2958
\end{verbatim}

\begin{verbatim}
otbcli_GridBasedImageResampling -io.in image2.tif
                                -io.out image2_epipolar.tif
                                -grid.in outimage2_grid.tif
                                -out.sizex 4464
                                -out.sizey 2958
\end{verbatim}

As you can see, we set sizex and sizey parameters using output values given by
the \application{StereoRectificationGridGenerator} application to set the size
of the output epipolar images.

\subsubsection{Disparity estimation: Block matching on epipolar lines}

Finally, we can begin the stereo correspondences! Things become a little are
more complicated but lets describe now the powerfulness of the
\application{BlockMatching} application.  To correlate block from the first
image with a block in the second image, we've got two epipolar data where the
use of epipolar lines allows us to constrain the search along a 1-dimensional
line as opposed to the entire 2-dimensional image. Moreover, block matching is
used, as opposed to single point matching, because of the obvious advantage that
correlating blocks is much more likely to reflect a true match.

For each point in the first image (the ``baseline''), we can search for the
corresponding pixel in the second image and use the co-localisation function
describes in \ref{ssec:epipolar}.

An almost complete spectrum of stereo correspondence algorithms has been
published and it is still augmented at a significant rate!See for example
\href{http://en.wikipedia.org/wiki/Block-matching_algorithm} for example. The
\otb implements different strategies for block matching:

\begin{itemize}
\item Sum of Square Distances block-matching (SSD)
\item Normalized Cross-Correlation (NCC)
\item Lp pseudo-norm (LP)
\end{itemize}

An other important mandatory parameter of the application is the range of
disparities. In theory, the block matching can perform a blind exploration and
search for a infinite range of disparities between the stereo pair. We need now
to evaluate the range of disparities where the block (from the deepest point on
Earth, the Challenger Deep (\url{http://en.wikipedia.org/wiki/Challenger_Deep})
to the Everest summit!  I deliberately exaggerated but you can imagine that with
a smaller range you can imagine that the block matching algorithm can take a lot
of time.  That's why these parameters are mandatory for the application and as
consequence we need to estimate them manually. That's pretty simple using the
two epipolar images.

In my case, I take one point on a flat area. The image coordinate in $image_{1}$
is $[1970,1525]$ and in $image_{2}$ is $[1970,1526]$ I take after that a second
point on a higher region (in my case a point near the top of the Pyramid of
Cheops!). The image coordinate of this pixel in $image_{1}$ is $[1661,1299]$ and
in $image_{2}$ is $[1633,1300]$.  So you see for the vertical exploration, I
must set the minimum value to a minimum -30 (the convention for the sign of the
disparity range is from image1 to image2).

Note that, this estimation can be facilitate using an external DEM in the
\application{StereoRectificationGridGenerator} application.  opj devant toutes
les fonctions Concerning the vertical disparity, in the first step we said that
we reduce the problem of 3D extraction to a 1D problem, that's not completely
true in general cases. In our case, there are small disparities in the vertical
direction which are due to parallax errors (i.e. epipolar lines exhibit a small
shift in the vertical direction, around 1 pixel). So that, exploration in the
vertical direction of disparities are so typically smaller than horizontal
one. You can also estimate them on the epipolar couple (in my case I use a range
of -1 to 1).

One more time take care of the sign of this minimum and this maximum for
disparities (always from image1 to image2).

The command line for the \application{BlockMatching} application is :
\begin{verbatim}
otbcli_BlockMatching -io.inleft image1_epipolar.tif
                     -io.inright image2_epipolar.tif
                     -io.out disparity_map_ncc.tif
                     -bm.minhd -50
                     -bm.maxhd 20
                     -bm.minvd 1
                     -bm.maxvd 1
                     -mask.nodata 0
                     -mask.variancet 10
                     -ram 2048
                     -io.outmetric 1
                     -bm.metric ncc
\end{verbatim}

The application creates by default a two bands image : horizontal disparity and
vertical disparity.

The \application{BlockMatching} application gives access to a lot of other
powerful functionalities to improve the quality of the disparity.

Let's describe now these functionalities:

\begin{itemize}
\item -io.outmetric : Output the metric:if the optimal metric values image is
  activated, it will be concatenated it to the output image (which will then
  have three bands : horizontal disparity, vertical disparity and metric value)
\item -bm.subpixel : Perform sub-pixel estimation of disparities
\item -mask.nodata 0 : as a consequence, you can specify a no-data value which
  will discard pixels with this value (for example the epipolar geometry can
  generate large part of images with black pixel)
\item -mask.variancet : The block matching algorithm have difficulties to find
  matches on uniform zone. We can use the variance threshold to discard those
  regions and speed-up again computation time.
\end{itemize}

Of course all these parameters can be combine to improve the disparity map.

\subsubsection{From disparity to Digital Elevation Model}

With the previous application, we've evaluated disparities between images. The
next and last step is now to transform the disparity map in an elevation
information and produce an elevation map.  It uses as input the disparity map
(horizontal and vertical) to produce a Digital Elevation Model (DEM) with a
regular sampling. The elevation values is computed from the triangulation of the
"left-right" pairs of pixels matched and when several elevations are possible on
a DEM cell, the highest is kept.

The first important point is that its often a good idea to refine your disparity
map given by the \application{BlockMatching} application to only keep relevant
disparities. For this, we're goig theto use the output optimal metric image and
threshold disparities among this value.

For example, if you've used Normalized Cross-Correlation (NCC), you can only
keep disparities where optimal metric is superior to $0.9$. Other evaluated
disparities can be consider as innacurated and will not be used to compute
elevation information.

This refinement can be easily done with \app.

We use first the \application{BandMath} application to threshold disparity.
\begin{verbatim}
otbcli_BandMath -il disparity_map_ncc.tif 
                -out thres_hdisparity.tif 
                -exp "if(im1b3>00.9,im1b1,0)"
\end{verbatim}

\begin{verbatim}
otbcli_BandMath -il disparity_map_ncc.tif 
                -out thres_vdisparity.tif 
                -exp "if(im1b3>00.9,im1b2,0)"
\end{verbatim}

And then, concatenate thresholded disparities using the \application{ConcatenateImages}: 

\begin{verbatim}
otbcli_ConcatenateImages -il thres_hdisparity.tif  thres_vdisparity.tif  
                -out thres_hvdisparity.tif 
\end{verbatim}

\begin{verbatim}
otbcli_DisparityMapToElevationMap -io.in thres_hvdisparity.tif
                                  -io.left image1.tif
                                  -io.right image2.tif
                                  -io.lgrid outimage1_pyramid.tif
                                  -io.rgrid outimage2_pyramid.tif
                                  -io.out disparity_map_ssd_to_elevation.tif
                                  -hmin 14
                                  -hmax 230
                                  -elev.average.value 100
                                  -ram 2048
\end{verbatim}

It produces the elevation map on the ground area covered by the stereo pair.
