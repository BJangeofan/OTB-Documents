\section{Image processing and information extraction}\label{sec:improc}

\subsection{Simple calculus with channels}\label{ssec:calculus}

The \application{BandMath} application provides a simple and efficient
way to perform band operations. The command line application and the
corresponding Monteverdi module (shown in the section \ref{Band_math module})
are based on the same standards. It computes a band wise operation according
to a user defined mathematical expression. The following code computes the
absolute difference between first bands of two images:

\begin{verbatim}
otbcli_BandMath -il input_image_1 input_image_2
                -exp "abs(im1b1 - im2b1)"
                -out output_image
\end{verbatim}

The naming convention "im[x]b[y]" designates the yth band of the xth input image.

The \application{BandMath} application embeds built-in operators and functions
(listed \href{http://muparser.sourceforge.net/mup_features.html#idDef2}{here}),
allowing a vast choice of possible operations.

\subsection{Images with no-data values}\label{ssec:nodata}

Image files can contain a no-data value in their metadata. It represents a
special pixel value that should be treated as "no data available for this pixel".
For instance, SRTM tiles use a particular no-data value of -32768 (usually
found on sea areas).

On multiband images, the no-data values are handled independently for each band.
The case of an image with no-data values defined only for a subset of its bands
is supported.

This metadata is now handled by OTB image readers and writer (using the GDAL
driver). The no-data value can be read from an image files and stored in the
image metadata dictionary. It can also be exported by image writers. The OTB
filters that produce a no-data value are able to export this value so that the
output file will store it.

An application has been created to manage the no-data value. The
\application{ManageNoData} application has the following features :
\begin{itemize}
\item Build a mask corresponding to the no-data pixels in the input image : it gives you
a binary image of the no-data pixels in your input image.
\item Change the no-data value of the input image : it will change all pixels that carry
the old no-data value to the new one and update the metadata
\item Apply an external mask to the input image as no-data : all the pixels that
corresponds have a null mask value are flagged as no-data in the output image.
\end{itemize}

For instance, the following command converts the no-data value of the input
image to the default value for DEM (which is -32768) :
\begin{verbatim}
otbcli_ManageNoData -in input_image.tif
                    -out output_image.tif
                    -mode changevalue
                    -mode.changevalue.newv -32768
\end{verbatim}

The third mode "apply" can be useful if you apply a formula to the entire image.
This will likely change the values of pixels flagged as no-data, but the no-data
value in the image metadata doesn't change. If you want to
fix all no-data pixels to their original value, you can extract the mask of the
original image and apply it on the output image. For instance:
\begin{verbatim}
otbcli_ManageNoData -in input_image.tif
                    -out mask.tif
                    -mode buildmask

otbcli_BandMath -il input_image.tif
                -out filtered_image.tif
                -exp "2*im1b1-4"

otbcli_ManageNoData -in filtered_image.tif
                    -out output_image.tif
                    -mode apply
                    -mode.apply.mask mask.tif
\end{verbatim}

You can also use this "apply" mode with an additional parameter "mode.apply.ndval".
This parameter allow to set the output nodata value applying according to your input mask. 

\subsection{Segmentation}\label{ssec:segmentation}

Segmenting objects across a very high resolution scene and with a controlled
quality is a difficult task for which no method has reached a sufficient level
of performance to be considered as operational.

Even if we leave aside the question of segmentation quality and
consider that we have a method performing reasonably well on our data
and objects of interest, the task of scaling up segmentation to real
very high resolution data is itself challenging. First, we can not
load the whole data into memory, and there is a need for on the flow
processing which does not cope well with traditional segmentation
algorithms. Second, the result of the segmentation process
itself is difficult to represent and manipulate efficiently.

The experience of segmenting large remote sensing images is packed into a single
\application{Segmentation} in \app.

You can find more information about this application
\href{http://blog.orfeo-toolbox.org/preview/coming-next-large-scale-segmentation}{here}.


%\subsection{Segmentation}\label{ssec:segmentation}
%todo.

%\subsection{Change detection}\label{ssec:changedetection}
%todo.

%\subsection{Object-based image analysis}\label{ssec:obia}
%todo.

\subsection{Large-Scale Mean-Shift (LSMS) segmentation}

LSMS is a segmentation workflow which allows to perform tile-wise
segmentation of very large image with theoretical guarantees of
getting identical results to those without tiling. 

It has been developed by David Youssefi and Julien Michel during David internship
at CNES.

For more a complete description of the LSMS method, please refer to the following publication, \textit{J. Michel,
D. Youssefi and M. Grizonnet, "Stable Mean-Shift Algorithm and Its Application
to the Segmentation of Arbitrarily Large Remote Sensing Images," in IEEE
Transactions on Geoscience and Remote Sensing, vol. 53, no. 2, pp. 952-964,
Feb. 2015.}

The workflow consists in chaining 3 or 4 dedicated applications and
produces a GIS vector file with artifact-free polygons corresponding
to the segmented image, as well as mean and variance of the radiometry
of each band for each polygon.

\subsubsection{Step 1: Mean-Shift Smoothing}

The first step of the workflow is to perform Mean-Shift smoothing with
the \application{MeanShiftSmoothing} application:

\begin{verbatim}
otbcli_MeanShiftSmoothing -in input_image 
                          -fout filtered_range.tif 
                          -foutpos filtered_spat.tif 
                          -ranger 30 
                          -spatialr 5 
                          -maxiter 10 
                          -modesearch 0
\end{verbatim}

Note that the \emph{modesearch} option should be disabled, and
that the \emph{foutpos} parameter is optional: it can be activated
if you want to perform the segmentation based on both spatial and
range modes.

This application will smooth large images by streaming them, and
deactivating the \emph{modesearch} will guarantee that the results
will not depend on the streaming scheme. Please also note that the
\emph{maxiter} is used to set the margin to ensure these
identical results, and as such increasing the \emph{maxiter} may
have an additional impact on processing time.

\subsubsection{Step 2: Segmentation}

The next step is to produce an initial segmentation based on the
smoothed images produced by the \application{MeanShiftSmoothing}
application. To do so, the \application{LSMSSegmentation} will process
them by tiles whose dimensions are defined by the
\emph{tilesizex} and \emph{tilesizey} parameters, and by
writing intermediate images to disk, thus keeping the memory
consumption very low throughout the process. The segmentation will
group together adjacent pixels whose range distance is below the
\emph{ranger} parameter and (optionally) spatial distance is
below the \emph{spatialr} parameter.

\begin{verbatim}
otbcli_LSMSSegmentation -in filtered_range.tif
                        -inpos filtered_spatial.tif
                        -out  segmentation.tif uint32 
                        -ranger 30 
                        -spatialr 5 
                        -minsize 0 
                        -tilesizex 256 
                        -tilesizey 256
\end{verbatim}

Note that the final segmentation image may contains a very large
number of segments, and the \emph{uint32} image type should
therefore be used to ensure that there will be enough labels to index
those segments. The \emph{minsize} parameter will filter segments
whose size in pixels is below its value, and their labels will be set
to 0 (nodata). 

Please note that the output segmented image may look patchy, as if
there were tiling artifacts: this is because segments are numbered
sequentially with respect to the order in which tiles are
processed. You will see after the result of the vectorization step
that there are no artifacts in the results.

The \application{LSMSSegmentation} application will write as many
intermediate files as tiles needed during processing. As such, it may
require twice as free disk space as the final size of the final
image. The \emph{cleanup} option (active by default) will clear the
intermediate files during the processing as soon as they are not
needed anymore. By default, files will be written to the current
directory. The \emph{tmpdir} option allows to specify a different
directory for these intermediate files.

\subsubsection{Step 3 (optional): Merging small regions}

The \application{LSMSSegmentation} application allows to filter out
small segments. In the output segmented image, those segments will be
removed and replaced by the background label (0). Another solution to
deal with the small regions is to merge them with the closest big
enough adjacent region in terms of radiometry. This is handled by the
\application{LSMSSmallRegionsMerging} application, which will output a
segmented image where small regions have been merged. Again, the
\emph{uint32} image type is advised for this output image.

\begin{verbatim}
otbcli_LSMSSmallRegionsMerging -in filtered_range.tif
                               -inseg segementation.tif
                               -out segmentation_merged.tif uint32 
                               -minsize 10 
                               -tilesizex 256 
                               -tilesizey 256
\end{verbatim}

The \emph{minsize} parameter allows to specify the threshold on the
size of the regions to be merged. Like the \application{LSMSSegmentation}
application, this application will process the input images tile-wise
to keep resources usage low, with the guarantee of identical
results. You can set the tile size using the \emph{tilesizex} and
\emph{tilesizey} parameters. However unlike the
\application{LSMSSegmentation} application, it does not require to
write any temporary file to disk.

\subsubsection{Step 4: Vectorization}

The last step of the LSMS workflow consists in the vectorization of the
segmented image into a GIS vector file. This vector file will contain
one polygon per segment, and each of these polygons will hold
additional attributes denoting the label of the original segment, the
size of the segment in pixels, and the mean and variance of each band
over the segment. The projection of the output GIS vector file will be
the same as the projection from the input image (if input image has no
projection, so does the output GIS file).

\begin{verbatim}
otbcli_LSMSVectorization -in input_image 
                         -inseg segmentation_merged.tif 
                         -out segmentation_merged.shp 
                         -tilesizex 256 
                         -tilesizey 256
\end{verbatim}

This application will process the input images tile-wise
to keep resources usage low, with the guarantee of identical
results. You can set the tile size using the \emph{tilesizex} and
\emph{tilesizey} parameters. However unlike the
\application{LSMSSegmentation} application, it does not require to
write any temporary file to disk.

\subsection{Dempster Shafer based Classifier Fusion}\label{ssec:classifierfusion}

This framework is dedicated to perform cartographic validation starting
from the result of a detection (for example a road extraction), enhance
the results fiability by using a classifier fusion algorithm. Using a
set of descriptor, the processing chain validates or invalidates the
input geometrical features.

% \subsubsection{Prequel: Road Extraction}
%
% The first step of this recipe is to produce an interesting and adapted
% input. The \application{otbRoadExtractionApplication},  included in
% \app , provides a set of geometrical features that can be used as input of the following process. This is only an example, the Dempster-Shafer framework was not designed specifically to be used with otbRoadExtractionApplication but it is a good example of what the input should be like.

\subsubsection{Fuzzy Model (requisite)}

The \application{DSFuzzyModelEstimation} application performs the fuzzy
model estimation (once by use case: descriptor set / Belief support /
Plausibility support). It has the following input parameters :
\begin{itemize}
\item \verb?-psin? a vector data of positive samples enriched according to the
"Compute Descriptors" part
\item \verb?-nsin? a vector data of negative samples enriched according to the
"Compute Descriptors" part
\item \verb?-belsup? a support for the Belief computation
\item \verb?-plasup? a support for the Plausibility computation
\item \verb?-desclist? an initialization model (xml file) or a descriptor name list
(listing the descriptors to be included in the model)
\end{itemize}

The application can be used like this:
\begin{verbatim}
otbcli_DSFuzzyModelEstimation -psin     PosSamples.shp
                              -nsin     NegSamples.shp
                              -belsup   "ROADSA"
                              -plasup   "NONDVI" "ROADSA" "NOBUIL"
                              -desclist "NONDVI" "ROADSA" "NOBUIL"
                              -out      FuzzyModel.xml
\end{verbatim}

The output file \verb?FuzzyModel.xml? contains the optimal model to perform
informations fusion.

\subsubsection{First Step: Compute Descriptors}

The first step in the classifier fusion based validation is to compute, for
each studied polyline, the choosen descriptors. In this context, the
\application{ComputePolylineFeatureFromImage} application can be used for a
large range of descriptors. It has the following inputs :
\begin{itemize}
\item \verb?-in? an image (of the sudied scene) corresponding to the choosen
descriptor (NDVI, building Mask\dots)
\item \verb?-vd? a vector data containing polyline of interest
\item \verb?-expr? a formula ("b1 \textgreater 0.4", "b1 == 0") where b1 is
the standard name of input image first band
\item \verb?-field? a field name corresponding to the descriptor codename
(NONDVI, ROADSA...)
\end{itemize}

The output is a vector data containing polylines with a new field containing
the descriptor value. In order to add the "NONDVI" descriptor to an input
vector data ("inVD.shp") corresponding to the percentage of pixels along a
polyline that verifies the formula "NDVI \textgreater 0.4" :

\begin{verbatim}
otbcli_ComputePolylineFeatureFromImage -in   NDVI.TIF
                                       -vd  inVD.shp
                                       -expr  "b1 > 0.4"
                                       -field "NONDVI"
                                       -out   VD_NONDVI.shp
\end{verbatim}

\verb?NDVI.TIF? is the NDVI mono band image of the studied scene.
This step must be repeated for each choosen descriptor:

\begin{verbatim}
otbcli_ComputePolylineFeatureFromImage -in   roadSpectralAngle.TIF
                                       -vd  VD_NONDVI.shp
                                       -expr  "b1 > 0.24"
                                       -field "ROADSA"
                                       -out   VD_NONDVI_ROADSA.shp
\end{verbatim}

\begin{verbatim}
otbcli_ComputePolylineFeatureFromImage -in   Buildings.TIF
                                       -vd  VD_NONDVI_ROADSA.shp
                                       -expr  "b1 == 0"
                                       -field "NOBUILDING"
                                       -out   VD_NONDVI_ROADSA_NOBUIL.shp
\end{verbatim}

Both \verb?NDVI.TIF? and \verb?roadSpectralAngle.TIF? can be produced
using \mont feature extraction capabilities, and \verb?Buildings.TIF?
can be generated using \mont rasterization module. From now on,
\verb?VD_NONDVI_ROADSA_NOBUIL.shp? contains three descriptor fields.
It will be used in the following part.

\subsubsection{Second Step: Feature Validation}

The final application (\application{VectorDataDSValidation}) will
validate or unvalidate the studied samples using
\href{http://en.wikipedia.org/wiki/Dempster\%E2\%80\%93Shafer_theory}{the Dempster-Shafer theory}
. Its inputs are :
\begin{itemize}
\item \verb?-in? an enriched vector data "VD\_NONDVI\_ROADSA\_NOBUIL.shp"
\item \verb?-belsup? a support for the Belief computation
\item \verb?-plasup? a support for the Plausibility computation
\item \verb?-descmod? a fuzzy model FuzzyModel.xml
\end{itemize}
The output is a vector data containing only the validated samples.

\begin{verbatim}
otbcli_VectorDataDSValidation -in      extractedRoads_enriched.shp
                              -descmod FuzzyModel.xml
                              -out     validatedSamples.shp
\end{verbatim}
