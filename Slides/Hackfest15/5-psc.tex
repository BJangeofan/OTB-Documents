%----------------------------------------------------------------------------------------
%	PACKAGES & THEMES
%----------------------------------------------------------------------------------------

\documentclass[8pt]{beamer}

\usepackage{etex}
\mode<presentation> {

\usetheme{Vilanova}
}



\usepackage[english]{babel}
\usepackage[utf8]{inputenc}
\usepackage{array}
\usepackage{chronology}
\let\CHRONOLOGY\chronology
\let\endCHRONOLOGY\endchronology
\def\chronology{\shorthandoff{;}\CHRONOLOGY}
\def\endchronology{\endCHRONOLOGY\shorthandon{;}}
\usepackage{pstricks}
\usepackage{graphicx}
\usepackage{booktabs}
\usepackage{amsmath,amssymb,amsthm}
\usepackage{xcolor}
\usepackage{textpos}
\usepackage{tikz}
\usepackage{xmpincl}
\usetikzlibrary{arrows}
\usepackage{pifont}

\usepackage{listings,color}

\definecolor{listcomment}{rgb}{0.0,0.5,0.0}
\definecolor{listkeyword}{rgb}{0.0,0.0,0.5}
\definecolor{listnumbers}{gray}{0.65}
\definecolor{listlightgray}{gray}{0.955}
\definecolor{listwhite}{gray}{1.0}


%% \setbeamertemplate{background canvas}{\includegraphics
%%    [width=\paperwidth,height=\paperheight]{./images/title.pdf}}

\AtBeginSection[]
{
\addtocounter{framenumber}{-1}
\begin{frame}
\frametitle{Sommaire}
\tableofcontents[currentsection]
\end{frame}}

%----------------------------------------------------------------------------------------
%	PAGE TITRE
%----------------------------------------------------------------------------------------
\title{Steering OTB: an introduction to the new OTB PSC}
\includexmp{images/cc}
\subtitle{High level guidance and coordination for the ORFEO ToolBox}
\author{OTB development team}% date and event here
\date{3 - 5 june 2015, Toulouse}

\pgfdeclareimage[height=96mm,width=128mm]{background}{images/fondsClairSansLogo}
\pgfdeclareimage[height=0.2cm]{cc}{images/CC-licence.png}
\setbeamertemplate{background}{\pgfuseimage{background}}
\pgfdeclareimage[height=0.6cm]{logoIncrust}{images/logoIncrust}
\logo{
\begin{tabular}{p{0.22\textwidth}p{0.58\textwidth}p{0.1\textwidth}p{0.1\textwidth}}
\href{http://creativecommons.org/licenses/by-sa/3.0/}{\pgfuseimage{cc}}
& \vspace{-0.03\textwidth} \scriptsize{} % date and event here
&  & \href{http://www.orfeo-toolbox.org}{\pgfuseimage{logoIncrust}}\\
\end{tabular}
}

\begin{document}
\begin{frame}
\titlepage
\end{frame}

\begin{frame}
\frametitle{Introduction}
The aim of this presentation is:
\begin{itemize}
\item To describe how Orfeo ToolBox project worked until PSC creation
\item To explain what the PSC is and how it works
\item To sketch the possibilities offered by this new, more open governance
\end{itemize}

\end{frame}

\begin{frame}
\frametitle{How OTB worked before PSC: the benevolent dictatorship dynasty}
\begin{block}{Who makes feature requests}
\begin{itemize}
\item Users from Orfeo CNES program (main funding source for 8 years)
\item CNES team (based on feedback from ml and orfeo)
\item Users from mailing list
\end{itemize}
\end{block}

\begin{block}{Who decides}
\begin{itemize}
\item CNES team (b.d.: Jordi, then Jordi + Emmanuel, then Jordi + Manuel, then Julien + Manuel)
\item With the support of CS dev team
\end{itemize}
\end{block}

\begin{block}{Who actually writes code}
\begin{itemize}
\item CS dev team (funded by CNES, re-conducted through 4 consecutive call for tenders) at 75\%
\item CNES team at 20\%
\item Contributors at 5\% (most often goes through CNES or CS dev team)
\end{itemize}
\end{block}

\textcolor{red}{So CNES gathers, decides and funds everything (hopefully with consideration for users and contributors)}

\end{frame}

\begin{frame}
\frametitle{The making of (a release of) Orfeo ToolBox}
\begin{enumerate}
\item CNES decides what major changes (features and infra) will be made
\item The dev team starts iterating scrum sprints (around 2 weeks each)
\item At each sprint end, CNES reviews with the dev team the next sprint and release scopes to accommodate urgent/new requests 
\item After 4 to 6 sprints (or when the scope is achieved to a satisfactory level), the Release Candidate is announced
\item If no major issue shows up, the Release Candidate becomes the final release
\end{enumerate}
\end{frame}

\begin{frame}
\frametitle{Ok, but this worked for 8 years, why changing now?}
\begin{block}{For transparency}
\begin{itemize}
\item Users do not know the mid-term directions of the project
\item Users are often informed afterward of major changes
\item No insight on motivations behind some decisions
\item Difficult to participate in decision making
\item OTB is a big project now, someone may want to get involved more deeply
\end{itemize}
\end{block}

\begin{block}{For more, easier contributions}
\begin{itemize}
\item Question: what is the process for contributing code to OTB? 
\item Question: how can I know if my contribution will be accepted?
\item Question: if I contribute a lot, do I get a grip on decision making?
\end{itemize}
\end{block}

\begin{block}{For sustainability}
\begin{itemize}
\item What if, one day, CNES stops funding OTB at the current level?
\item We need new actors to be able to get involved in OTB!
\end{itemize}
\end{block}
\end{frame}

\begin{frame}
\frametitle{Introducing Orfeo ToolBox Project Steering Committee (starts March 2015)}

\begin{block}{The PSC statement}
Verbatim from the \href{http://wiki.orfeo-toolbox.org/index.php/Project_Steering_Committee}{PSC status}:
\begin{itemize}
\item \textit{The aim of the OTB Project Steering committee (PSC) is to provide high level guidance and coordination for the ORFEO ToolBox.}
\item \textit{It provides a central point of contact for the project and arbitrates disputes. It is also a stable base of "institutional knowledge" to the project and tries its best to involve more developers.}
\item \textit{It should help to guarantee that OTB remains open and company neutral.}
\end{itemize}
\end{block}

\begin{block}{... and its scope}
\begin{itemize}
\item Roadmaps
\item Communication
\item Users support and documentation
\item Contribution management
\item Release planning
\item Handling of legal issues
\end{itemize}

\end{block}

\end{frame}

\begin{frame}
\frametitle{PSC Members and roles}

\begin{itemize}
\item All members have equal standing and voice in the PSC
\item The PSC seats are non-expiring
\item PSC members may resign their position, or be asked to vacate their seat after a unanimous vote of no confidence from the remaining PSC members
\item Members can be assigned roles corresponding to each category of the PSC scope
\end{itemize}

The expectations on PSC members are:
\begin{itemize}
\item Be willing to commit to the OTB development effort
\item Be responsive to requests for information from fellow members
\item Be able and willing to attend on-line meetings
\item Act in the best interests of the project 
\end{itemize}
\end{frame}

\begin{frame}
\frametitle{Decision making in PSC}

\begin{block}{When is a vote required?}
\begin{enumerate}
\item Request for changes
  \begin{itemize}
    \item Anything that could cause backward compatibility issues
    \item Adding substantial amounts of new code
    \item Changing inter-subsystem APIs, or objects
  \end{itemize}
\item Addition or removal of PSC members (including the selection of a new Chair)
\item Release process 
\item Changing PSC rules and processes
\item Anything else that might be controversial 
\end{enumerate}
\end{block}
\begin{block}{Voting process}
\begin{itemize}
\item Proposals are written up and submitted on the otb-developers mailing list for discussion and voting
\item Proposals are available for review at least 3 days before vote is closed
\item Anyone is encouraged to comment and vote, though ultimately only PSC members vote are counted
\item Vote are casted by +1/-1. Acceptance if at least +2 and no veto (-1)
\end{itemize}
\end{block}
\end{frame}

\begin{frame}
\frametitle{Current PSC members}
\textit{In March 2015, CNES nominated 3 persons deeply involved in OTB as initial PSC members. They are responsible for defining PSC rules and establishing a fully functioning PSC.}
\vspace{0.5cm}
\begin{small}
\begin{tabular}{|l|l|l|l|}
\hline
\textbf{Name} 	&\textbf{Affiliation}	&\textbf{Role}\\
\hline
Manuel Grizonnet (chair) &	CNES  &	Infrastructure, release planning, legal issues\\
Jordi Inglada 	&CNES/CESBIO 		&User support and documentation, roadmaps\\
Julien Michel 	&CNES &	Communication, contributions \\
\hline
\end{tabular}
\end{small}

\begin{block}{?!?! ... but wait!}
\begin{itemize}
\item \textcolor{red}{This is (almost) the full dynasty of benevolent dictactors!}
\item Yes, but ...
\end{itemize}
\end{block}
\end{frame}

\begin{frame}
\frametitle{... There are new possibilities! (1/2)}

\begin{block}{You can become a PSC member}
\begin{itemize}
\item Anyone showing a substantial and ongoing involvement in OTB is eligible to be nominated to the OTB PSC
\item The PSC is not only composed of OTB developers as there are many ways to join and contribute to the project
\item Remember: an active membership will take time and effort
\item Note that the PSC is not a legal entity!
\end{itemize}
\end{block}

\begin{block}{You can submit RFCs}
\begin{itemize}
\item If you have an important contribution you want to make, you can submit a RFC
\item It will be discussed, decided, and logged publicly
\item You will be able to discuss a target release for the contribution to be included
\item Remember: contributions $\neq$ feature requests!
\item An alternate way for contributions exists: remote modules (see dedicated presentation)
\end{itemize}
\end{block}
\end{frame}

\begin{frame}
\frametitle{... There are new possibilities! (2/2)}

\begin{block}{You can comment on RFCs}
\begin{itemize}
\item You are encouraged to comment every RFC you want
\item ... or even vote!
\item Ultimately only PSC members gets their vote counted, their
  decision shall reflect the opinions from all participants
\end{itemize}
\end{block}

\begin{block}{You can submit feature requests}
One member of the PSC is responsible for roadmaps and will receive and track feature requests
\end{block}

\begin{block}{You know what is going on}
New RFCs, releases, new PSC members, status \ldots everything is discussed and logged publicly
\end{block}
\end{frame}

\begin{frame}
\frametitle{Final thoughts}

\begin{itemize}
\item The PSC is young (the entity, not its members ...)
\item It is a tool that can be adapted to best serve the interest of Orfeo ToolBox
\item Anything can be discussed and modified: processes, scope, rules, members\ldots
\item We hope to be more than 3 members in the future!
\end{itemize}

\end{frame}

\end{document}
