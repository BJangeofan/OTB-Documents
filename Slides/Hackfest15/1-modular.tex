%----------------------------------------------------------------------------------------
%	PACKAGES & THEMES
%----------------------------------------------------------------------------------------

\documentclass[8pt]{beamer}

\usepackage{etex}
\mode<presentation> {

\usetheme{Vilanova}
}



\usepackage[english]{babel}
\usepackage[utf8]{inputenc}
\usepackage{array}
\usepackage{pstricks}
\usepackage{graphicx}
\usepackage{booktabs}
\usepackage{amsmath,amssymb,amsthm}
\usepackage{xcolor}
\usepackage{textpos}
\usepackage{tikz}
\usepackage{xmpincl}
\usetikzlibrary{arrows}
\usepackage{pifont}



\usepackage{listings,color}

\definecolor{listcomment}{rgb}{0.0,0.5,0.0}
\definecolor{listkeyword}{rgb}{0.0,0.0,0.5}
\definecolor{listnumbers}{gray}{0.65}
\definecolor{listlightgray}{gray}{0.955}
\definecolor{listwhite}{gray}{1.0}


%% \setbeamertemplate{background canvas}{\includegraphics
%%    [width=\paperwidth,height=\paperheight]{./images/title.pdf}}

\AtBeginSection[]
{
\addtocounter{framenumber}{-1}
\begin{frame}
\frametitle{Sommaire}
\tableofcontents[currentsection]
\end{frame}}

%----------------------------------------------------------------------------------------
%	PAGE TITRE
%----------------------------------------------------------------------------------------
\title{Orfeo ToolBox users meeting and hackfest 2015}
\includexmp{images/cc}
\subtitle{The new modular build system of OTB 5}
\author{OTB development team}% date and event here
\date{3 - 5 june 2015, Toulouse}

\pgfdeclareimage[height=96mm,width=128mm]{background}{images/fondsClairSansLogo}
\pgfdeclareimage[height=0.2cm]{cc}{images/CC-licence.png}
\setbeamertemplate{background}{\pgfuseimage{background}}
\pgfdeclareimage[height=0.6cm]{logoIncrust}{images/logoIncrust}
\logo{
\begin{tabular}{p{0.22\textwidth}p{0.58\textwidth}p{0.1\textwidth}p{0.1\textwidth}}
\href{http://creativecommons.org/licenses/by-sa/3.0/}{\pgfuseimage{cc}}
& \vspace{-0.03\textwidth} \scriptsize{} % date and event here
&  & \href{http://www.orfeo-toolbox.org}{\pgfuseimage{logoIncrust}}\\
\end{tabular}
}

\begin{document}
\begin{frame}
\titlepage
\end{frame}

\begin{frame}
\frametitle{Motivations : the status with OTB <= 4.0}
\begin{itemize}
\item Code in directories but no real build dependency (cyclic dependencies between directories)
\item Third parties : a maintainance hell (activation/deactivation, external/internal, interdependencies)
\item Build system written in old CMake times (~ version 2.4), stayed more or less the same
\end{itemize}
\end{frame}

\begin{frame}
\frametitle{Design choices}
\begin{itemize}
\item Let's do it the ITK way : they faced the same issue a few years ago. The CMake gurus did the work
\item State-of-the-art renewed CMake code : a bunch of new features, and ready for later improvements
\item No more third party code inside OTB code (to the maximum extent...)
\end{itemize}
\end{frame}


\begin{frame}
\frametitle{What modular means ?}
\begin{itemize}
\item The OTB classes are organized in self-contained directories called modules
\item One module contain headers, definitions, tests, applications, ... for all classes of the module
\item Dependency tracking between modules is at the heart of the system : enable a module and its dependencies are enabled automatically
\item Only build what you want/use
\item Third parties are also modules now
\item Modules are organized in thematic Groups (no real build consequences here)
\end{itemize}
\end{frame}

\begin{frame}
\frametitle{How do I build OTB now?}
\begin{itemize}
\item Still with CMake, but way easier.
\item Less CMake options visible (unless you switch to advanced mode)
\item Better option naming, systematic : OTB\_USE\_XXX for third parties
\item Third parties are always external now. They can only be activated or not.
\item A default set of modules is defined
\end{itemize}
\end{frame}

\begin{frame}
\frametitle{The internals}

\begin{itemize}
\item Module enablement evaluated in two passes
\begin{itemize}
\item Modules which are activated
\item Third party deactivated, with all its dependees 
\end{itemize}
\end{itemize}


\begin{itemize}
\item BUILD\_DEFAULT\_MODULE
\begin{itemize}
\item ON = all modules activated, except Mapnik and its dependees
\item OFF = you choose each enabled module with the Module\_XXX option
\end{itemize}

\item OTB\_USE\_XXX 
\begin{itemize}
\item Only for third parties
\item Easy way to disable a third party and all its dependees
\item Is evaluated after the first
\end{itemize}

\end{itemize}
\end{frame}


\begin{frame}
\frametitle{What does a module look like ?}
A directory in the Modules subdir :
\begin{itemize}
\item otb-module.cmake and CMakeLists.txt file
\item include subdir : for .hxx files
\item src subdir : for .cxx files and a CMakeLists.txt. headers in here are not installed
\item test subdir : for tests
\item app subdir : for applications
\end{itemize}

See the Remote Module presentation for building a module from scratch in 5 minutes.
\end{frame}


\begin{frame}
\frametitle{Hey wait !}
\begin{itemize}
\item I don't care about your dependency hell, just build me the damned thing : Superbuild (in a few minutes)
\item Can I reuse this for my work : Remote modules  (in a few minutes)
\end{itemize}
\end{frame}

\end{document}
