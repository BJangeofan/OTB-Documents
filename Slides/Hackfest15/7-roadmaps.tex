% Created 2015-06-03 mer. 10:43
\documentclass[8pt]{beamer}
\usepackage[utf8]{inputenc}
\usepackage[T1]{fontenc}
\usepackage{fixltx2e}
\usepackage{graphicx}
\usepackage{longtable}
\usepackage{float}
\usepackage{wrapfig}
\usepackage{rotating}
\usepackage[normalem]{ulem}
\usepackage{amsmath}
\usepackage{textcomp}
\usepackage{marvosym}
\usepackage{wasysym}
\usepackage{amssymb}
\usepackage{hyperref}
\tolerance=1000
\usepackage{etex}
\mode<presentation>{\usetheme{Vilanova}}
\usepackage[english]{babel}
\usepackage[utf8]{inputenc}
\usepackage{array}
\usepackage{chronology}
\let\CHRONOLOGY\chronology
\let\endCHRONOLOGY\endchronology
\def\chronology{\shorthandoff{;}\CHRONOLOGY}
\def\endchronology{\endCHRONOLOGY\shorthandon{;}}
\usepackage{pstricks}
\usepackage{graphicx}
\usepackage{booktabs}
\usepackage{amsmath,amssymb,amsthm}
\usepackage{xcolor}
\usepackage{textpos}
\usepackage{tikz}
\usepackage{xmpincl}
\usetikzlibrary{arrows}
\usepackage{pifont}
\usepackage{listings,color}
\definecolor{listcomment}{rgb}{0.0,0.5,0.0}
\definecolor{listkeyword}{rgb}{0.0,0.0,0.5}
\definecolor{listnumbers}{gray}{0.65}
\definecolor{listlightgray}{gray}{0.955}
\definecolor{listwhite}{gray}{1.0}
\AtBeginSection[]{\addtocounter{framenumber}{-1}\begin{frame}\frametitle{Outline}\tableofcontents[currentsection]\end{frame}}
\includexmp{images/cc}
\subtitle{What's ahead: An overview of OTB road-maps}
\pgfdeclareimage[height=96mm,width=128mm]{background}{images/fondsClairSansLogo}
\pgfdeclareimage[height=0.2cm]{cc}{images/CC-licence.png}
\setbeamertemplate{background}{\pgfuseimage{background}}
\pgfdeclareimage[height=0.6cm]{logoIncrust}{images/logoIncrust}
\logo{  \begin{tabular}{p{0.22\textwidth}p{0.58\textwidth}p{0.1\textwidth}p{0.1\textwidth}}   \href{http://creativecommons.org/licenses/by-sa/3.0/}{\pgfuseimage{cc}}    & \vspace{-0.03\textwidth} \scriptsize{}     &  & \href{http://www.orfeo-toolbox.org}{\pgfuseimage{logoIncrust}}\\\end{tabular} }
\usetheme{default}
\author{OTB development team}
\date{3 - 5 june 2015, Toulouse}
\title{Orfeo ToolBox users meeting and hackfest 2015}
\hypersetup{
  pdfkeywords={},
  pdfsubject={},
  pdfcreator={Emacs 24.3.1 (Org mode 8.2.4)}}
\begin{document}

\maketitle


\begin{frame}[label=sec-1]{ORFEO: the original OTB road-map}
\begin{itemize}
\item OTB was initially a means to prepare the use of the imagery provided
by the ORFEO (Pléiades HR + Cosmo Skymed) programme
\item OTB was meant to be a receptacle to capitalise the results of R\&D
studies funded by CNES
\item The road-map was only about \uline{methods} for the image \uline{information
extraction} pipeline
\item The road-map was fed as follows
\begin{enumerate}
\item Gather thematic users' needs
\item Factor needs in terms of processing blocks
\item Identify priorities
\begin{itemize}
\item technical readiness level (availability as software, literature, etc.)
\item impact on users' needs
\item cost of implementation
\end{itemize}
\end{enumerate}
\end{itemize}
\end{frame}
\begin{frame}[label=sec-2]{Themes in the first road-map}
\begin{itemize}
\item Pre-processing : geometry, radiometry
\item 2D information extraction (surfaces, networks, etc.)
\item 3D information extraction
\item Object detection, recognition and identification
\item Multi-temporal analysis and change detection
\item Image fusion and synthesis
\item Product delivery and display (GIS formats, visualisation)
\end{itemize}
\end{frame}

\begin{frame}[label=sec-3]{Example of item in the road-map}
\begin{center}
\begin{tabular}{ll}
Sujet & 6.1 Détection de changements : approches objet\\
\hline
Descriptif & Le problème est ici de détecter des changement\\
 & abrupts entre deux images acquises à des dates différentes\\
 & sans avoir besoin d'extraire des informations 3D,\\
 & mais en essayant d'exploiter des informations extraites\\
 & à partir des objets contenus dans les images.\\
\hline
Missions visées & Pléiades HR\\
\hline
Priorité & Haute\\
\hline
Contexte de réalisation & Orfeo Toolbox + études internes (stage V. Poulain) +\\
 & bourse de thèse (V. Poulain)\\
\hline
\end{tabular}
\end{center}
\end{frame}

\begin{frame}[label=sec-4]{Need for a new road-map}
\begin{itemize}
\item The ORFEO programme has come to an end
\item Upcoming EO missions
\begin{itemize}
\item Sentinels and LANDSAT8
\item SPOT6,7
\item SWOT
\item others \ldots{}
\end{itemize}
\item OTB is widely used beyond CNES and its neighbourhood
\begin{itemize}
\item Need to take into account other long-term goals
\end{itemize}
\item OTB has become much more than a library for R\&D
\begin{itemize}
\item Increased pressure on performance and genericity
\item Not 1 but 3 road-maps: features + technical + infrastructure
\end{itemize}
\end{itemize}
\end{frame}
\begin{frame}[label=sec-5]{Interlude: OTB's Mission Statement proposal}
\begin{quote}
Provide a Free Software end-to-end solution for the Earth Observation
image information extraction pipeline based on a generic, high
performance C++ library upon which applications and processing chains
can be built to fulfil the needs of users from ground segment
processing chains to single user desktop applications.
\end{quote}
\end{frame}
\begin{frame}[label=sec-6]{What is a road-map for?}
\begin{itemize}
\item Declination of the Mission Statement down to concrete milestones
\item Communicates the project plan to users and stakeholders
\item Sets priorities but it is not static
\begin{itemize}
\item Can and must be updated as needs evolve, resources become
available or disappear, opportunities emerge
\end{itemize}
\item Organises ideas:
\begin{itemize}
\item feature requests, evil plans, elephants in the room
\end{itemize}
\item Difference between road-map and objectives of a release : mid-term
vs long-term
\end{itemize}
\end{frame}

\begin{frame}[fragile,label=sec-7]{The 3 OTB road-maps}
 \begin{enumerate}
\item Features
\begin{itemize}
\item Library
\begin{itemize}
\item ex: Sentinels sensor support, OBIA, data assimilation
\end{itemize}
\item Applications
\begin{itemize}
\item ex: classification framework
\end{itemize}
\item Monteverdi
\end{itemize}
\item Technical
\begin{itemize}
\item code optimisation and refactoring
\item developer guidelines
\begin{itemize}
\item ex: use STL or BOOST before coding anything, no naked \verb~new/delete~
\end{itemize}
\item target platforms (OS, compilers)
\item language bindings
\item external dependencies
\end{itemize}
\item Infrastructure
\begin{itemize}
\item CMS, Wiki
\item Dashboard
\item Build/packaging resources
\end{itemize}
\end{enumerate}
\end{frame}
\begin{frame}[label=sec-8]{How to efficiently maintain and use the road-maps}
\begin{itemize}
\item Maintaining
\begin{itemize}
\item A PSC member is responsible for
\begin{itemize}
\item keeping track of suggestions made trough the bug tracker
\item factoring suggestions into the appropriate items of the road-maps
\item presenting the updated road-maps to the PSC
\end{itemize}
\item PSC members with \href{http://wiki.orfeo-toolbox.org/index.php/Project_Steering_Committee#Current_members_and_roles}{roles} feed the road-maps with their needs
\item The full PSC sets the priorities and validates the road-maps
\end{itemize}
\item Using
\begin{itemize}
\item A release content is guided by the Feature and Technical road-maps
\item The PSC should
\begin{itemize}
\item check that release contents are in tune with the road-maps
\item identify the means needed to meet the road-maps' milestones
\end{itemize}
\end{itemize}
\end{itemize}
\end{frame}
\begin{frame}[label=sec-9]{Contributions to the road-maps}
\begin{itemize}
\item How ideas are collected currently
\begin{itemize}
\item The \href{http://wiki.orfeo-toolbox.org/index.php/OTB_Backlog}{backlog page} on the OTB wiki
\item The \href{http://scrum.orfeo-toolbox.org/}{OTB Jira}
\item The \href{https://bugs.orfeo-toolbox.org/my_view_page.php}{OTB bug tracker}
\end{itemize}
\item From now on
\begin{itemize}
\item The \href{https://bugs.orfeo-toolbox.org/my_view_page.php}{OTB bug tracker}
\item The \href{http://wiki.orfeo-toolbox.org/index.php/Project_Steering_Committee#Requests_for_changes}{RFC mechanism}
\end{itemize}
\item The road-maps will be publicly available on dedicated pages of the
OTB web site
\begin{itemize}
\item Not on the wiki, since they have to be validated by the PSC
\end{itemize}
\end{itemize}
\end{frame}
% Emacs 24.3.1 (Org mode 8.2.4)
\end{document}
