%% josis.tex 1.2   2009-06-11    JoSIS latex template 
%------------------------------------------------------------------
% Filename: josis.tex
%
% This file is intended as a template for typesetting articles for the
%
%                        Journal of Spatial Information Science.
%
% Please edit this template to generate your own formatted manuscripts 
% for submission to JOSIS. See http://josis.org for further details.
%
% The template was developed by Matt Duckham (http://www.duckham.org) 
% 

% Required documentclass definition for JOSIS
\documentclass{josis}
\usepackage{graphics}

% Article details for accepted manuscripts will be added by editorial staff 
%\josisdetails{%
%   volume=0, number=0, year=2009, firstpage=0, lastpage=1, 
%   received={January 1, 2009}, 
%   revised={March 1, 2009},
%   accepted={May 1, 2009},
%   published={June 1, 2009}
%}


% Add the running author and running title information
\runningauthor{\begin{minipage}{.9\textwidth}\centering Michel, Grizonnet, Jaen, Hermitte, Guinet, Harasse, Malik, Savinaud\end{minipage}}
\runningtitle{Large-scale segmentation of Very High Resolution satellite images using Orfeo ToolBox}

% Document begins
\begin{document}

% Insert your own title
\title{Large-scale segmentation of Very High Resolution satellite images using Orfeo ToolBox}

% Insert your manuscipts authors, affiliations, and addresses
\author{Julien Michel}
\author{Manuel Grizonnet}\affil{CNES, DCT/SI/AP, BPI 1219 18, avenue Edouard Belin, 31401 Toulouse Cedex 09 - France}
\author{Arnaud Jaen}
\author{Luc Hermitte}
\author{Jonathan Guinet}
\author{S\'ebastien Harasse}
\author{Julien Malik}
\author{Micka\"el Savinaud}\affil{CS Syst\`emes d'Information, Division ESPACE \& Renseignement - D\'epartement APPLICATIONS, Parc de la Grande Plaine - 5, Rue Brindejonc des Moulinais - BP 15872, 31506 Toulouse Cedex 05 - FRANCE}


\maketitle

% Add 5-10 keywords for every submission
\keywords{Open-source software, segmentation, remote sensing images, GIS}

% Add a short abstract of 150-250 words
\begin{abstract}
TODO
\end{abstract}


% Your main text begins here. 
\section{Introduction}

With the increase of the spatial resolution of satellite images,
analysis techniques such as Object Based Image Analysis or Spatial
Reasoning \cite{inglada2009qualitative} have become widely studied and
used. Because they use objects rather than pixels as their primitives,
these methods are very well adapted to represent and extract the
information contained in very high resolution imagery (VHR). Moreover,
reasoning on objects is often supported by very sound theories. Yet
one of their severe weakness is the process of obtaining the objects
themselves: segmentation is widely used as a pre-processing step for
these techniques, and it is well known that the task of segmenting all
categories of object of interest, across a large very high resolution
scene and with a controlled quality is a difficult task for which no
method has reached a sufficient level of performance to be considered
as operational.

Even if we leave aside the question of segmentation quality and
consider that we have a method performing reasonably well on our data
and objects of interest, the task of scaling up segmentation to real
very high resolution data is itself challenging. First, we can not
load the whole data into memory, and there is a need for on the flow
processing which does not cope well with traditional segmentation
algorithms \cite{shi2000normalized}. Second, the result of the
segmentation process itself is difficult to represent and manipulate
efficiently.

There are, to our best knowledge, few open source softwares able to
overcome these issues. In the frame of the development of the Orfeo
ToolBox \cite{}, we therefore initiated some work to provide software
components for this purpose.

\section{Data representation and conversion}

In this section, we review three standard structures to store
segmentation results, and explain why we selected the third one for
our framework.

The most common way of representing segmentation results from an image
processing perspective is to derive a raster where each pixel contains
the unique label of the segment it belongs to. Such a raster is also
known as label image. Yet this representation has numerous
drawbacks. First, accessing all pixels of a given segment identified
by its label requires parsing the whole label image. Second, storing
large segmentation results with billions of segments might require a
high number of bits to represent the label, thus increasing
dramatically the size of the output. Last, the constraint of label
uniqueness is very strong, yet not very useful: it is required since
label images only provides an implicit description of segments, based
on neighbouring pixels with same labels. This representation is of
limited interest for our purpose.

A second representation of segmented image which is further from image
processing is the map of label objects \cite{lehmann2008label}. A segment is
represented in Run Length Encoding (RLE), and all segments are indexed
by a unique label in a map. The RLE representation is more compact,
the map allows fast access to a segment given its label, and it also
allows to store attributes related to a segment, which is a first step
toward Object Based Image Analysis. The main drawback of this
representation is that there are no standard file format able to store
it on a file system, and it requires to convert either to or from
raster or vector representation at each read or write
operation. Moreover, this representation is neither really raster nor
really vector. Therefore, operations like walking pixels in the
segment or following its contour are more complex.

Neither of these two structures can fit our needs: the first one is
highly inefficient, while the second one lacks of existing file
formats. We chose to store our segmentation results as boundaries in a
vector structure and file format. This has numerous advantages. First,
it overcomes the issue of scaling up to large image segmentation: a
collection of vector objects, i.e. polygons or multi-polygons, is a
compact representation which will grow linearly with the size of the
image, and does not need an explicit unique indexing label. Second,
there are several file formats and even databases to represent this
kind of vector data, especially in the GIS world. Last, using such
formats guarantees a full interoperability between the segmentation
tools and most GIS software, which is desirable.

As described in the next section, we intend to derive a generic
framework for large scale remote sensing images segmentation, thus we
need to be compatible with the raster output of most segmentation
algorithms, which usually produce label image. For this purpose, we
need a component for exact conversion between the raster
representation and the vector one. e take advantage of the
polygonization and rasterization algorithms provided by GDAL
\cite{}. To handle on the flow input and output to vector data files
and databases, we then use the full extent of OGR functions
\cite{}. This abstraction layer allows our tool to address
simple file formats like ESRI shapefile or complex databases like
PostGIS seamlessly.

\section{A generic framework for large scale segmentation}

We define a segmentation algorithm in the Orfeo ToolBox as a filter
that accepts a remote sensing image as input and produces a label
image as output. This filter is not supposed to have streaming (on the
flow) capabilities, it is allowed to require the whole input image to
produce its output, and it can make an internal use of
multi-threading. Given such an algorithm, and a large image to segment
(for instance a basic Pleiades image \cite{} has an extent of 40
000 by 40 000 pixels), our framework works as follows:
1. Derive a tiling scheme according to the amount of memory available
on the computer and additional information such as file format
efficiency or user defined parameters,
2. For each tile of the tiling scheme:
   a. Load the corresponding image part into memory
   b. Segment the image extract with the segmentation algorithm
   c. Polygonize the result using a filter based on GDAL capabilities,
   d. Dump the polygons to file system or database through OGR
      abstraction.

The strength of this framework is that it will naturally scale up:
larger images will take more time to process and more space on file
system to store, but the system will never run out of the most limited
resource on most hardware, which is memory. The most time consuming
operation is the segmentation, and this is where we allow for parallel
processing depending on the segmentation algorithm implementation. For
instance, we implemented a multi-threaded version of the Mean shift
algorithm \cite{Comaniciu2002mean}.
 
A key feature is that we make few assumptions on the segmentation
algorithm, and that our framework will be able to perform with any
segmentation algorithm implemented under these assumptions. the same
code already runs with a basic connected components algorithm, two
different implementations of the Mean shift algorithm, and Watershed.

Of course, it has some drawbacks, the first one being that region
lying on tiles borders will be artificially split by the tiling
scheme. We address these issues in the following section.

\section{Pre and post-processing to enhance usability}

In order to get useful results from this large scale segmentation, we
need to address several issues. The first one issue is the splitting
of the regions on the border of tiles. The naive solution we
implemented is a simple stitching rule: we post-process the vector
data by looking for neighbour polygons lying on each side of a tile
border and merge them if their contact surface is large enough. More
sophisticated techniques might be derived in the future.

The second problem is that the polygons reflect exactly the shape of
the segment, and contain a large amount of vertices. This leads to
very heavy vector files, sometimes even heavier than the corresponding
raster image would be. To deal with this issue, we added
pre-processing as well as post-processing. First, we added an input
mask image to avoid segmenting unwanted regions, like no-data
pixels, clouds, or vegetation (if vegetation is not desired). We also
added a rule to remove very small segments, which are more likely to
correspond to segmentation noise than to real objects (this leads to
holes in the segmentation canvas). Last, we again used the
capabilities of OGR to perform a geometry simplification algorithm,
which simplify polygons by removing vertices according to a given
tolerance.

These additional processing allows to produce a cleaner vector
segmentation output which can then be used in a GIS software.


\section{Conclusion}

  This framework is only a first step toward providing an open source
 large scale OBIA and spatial reasoning framework. Future developments
 will include more stitching and tiling strategies to enhance the
 segmentation performance at tiles borders, and the computation of
 attributes along with polygons, such as statistics on radiometry from
 an image, or shape attributes. These attributes could then be used
 for reasoning or classification at the object level. While there is
 still a lot missing, we hope to provide a comprehensive environment
 for object-based techniques for VHR images analysis through the
 ongoing efforts to integrate Orfeo ToolBox within Quantum
 GIS \cite{}, a open source GIS software, via Sextante \cite{},
 and bridge the gap between remote sensing and GIS.


\subsubsection{References}

\bibliographystyle{acm}
\bibliography{refs}

\end{document}
