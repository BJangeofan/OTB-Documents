\chapter*{Contributors}
\noindent

The ORFEO Toolbox is a project conducted by CNES and developed in
cooperation with "Communication \&
  Syst\`{e}mes" (CS), \url{http://www.c-s.fr}.\\

This Software Guide is based on the ITK Software Guide: the build
process for the documentation, many examples and even the \LaTeX~ ~
sources were taken from ITK. We are very grateful to the ITK
developpers and contributors and especially to Luis Ib\'a\~nez.\\

The OTB specifics were implemented and documented by the OTB Development Team:
\begin{itemize}
  \item Jordi Inglada did most of the editing work for this guide and
  is guilty for the choice of data and examples. He also implemented
  the SVM classification approach and the change detection framework.
  \item Thomas Feuvrier is the OTB system guru: he implemented the
  build procedures for the code and the documentation; he implemented
  the IO functionalities and the streaming IO capabilities; he also
  developped the visualization tools.
  \item Julien Michel implemented the morphological pyramid
  functionnalities, the spatial reasoning tools, Kohonen's SOM,
  disparity map estimation, as
  well as some applications and filters.
\item Romain Garrigues is responsible for some applications and coded
  most of the ortho-rectification routines (ported from code
  developped by Miarintsoa Ramanantsimiavona); he also
  worked on the multi-platform installation procedures.
\item Emmanuel Christophe developped a road extraction algorithm and
  is responsible for the tutorials.
  \item Cyrille Valladeau coded some vegetation indices and ported the
  Bayesian fusion algorithm kindly provided by Julien Radoux (UCL).
  \item Gr\'egoire Mercier contributed the Kullback-Leibler change
  detectors, several SVM kernels and the SEM algorithm.
  \item Vincent Poulain contributed a DXF reader. 
  \item Patrick Imbo developped some filters and feature
  extraction algorithms.
  \item Caroline Ruffel coded the edge and line detectors among other
  fuctionnalities.



\end{itemize}

Contributions from users are expected and encouraged for the comming
versions of OTB.

