\chapter{Streaming and Threading}
\label{sec:StreamingAndThreading}

\index{Streaming}\index{Threading}

Streaming and threading are a complex issue in computing in general. This
chapter provides the keys to help you understand how it is working so you can
make the right choices later.

First, you have to be aware that streaming and threading are two different
things even if they are linked to a certain extent. In OTB:

\begin{itemize}
\item Streaming describes the ability to combine the processing of several
portion of a big image and to make the output identical as what you would have
gotten if the whole image was processed at once. Streaming is compulsory when
you're processing gigabyte images. 
\item Threading is the ability to process simultaneously different parts of the
image. Threading will give you some benefits only if you have a fairly recent
processor (dual, quad core and some older P4).
\end{itemize}



To sum up: streaming is good if you have big images, threading is good if you
have several processing units.

However, these two properties are not unrelated. Both rely on the filter ability
to process parts of the image and combine the result, that what the
ThreadedGenerateData() method can do.

For OTB, streaming is pipeline related while threading is filter related. If you
build a pipeline where one filter is not streamable, the whole pipeline is not
streamable: at one point, you would hold the entire image in memory. Whereas you
will benefit from a threaded filter even if the rest of the pipeline is made of
non-threadable filters (the processing time will be shorter for this particular
filter).


Even if you use a non streamed writer, each filter which has a
ThreadedGenerateData() will split the image into two and send each part to one
thread and you will notice two calls to the function.

If you have some particular requirement and want to use only one thread, you can
call the SetNumberOfThreads() method on each of your filter. 

%  TODO
% Add details on the splitting strategies...
%

