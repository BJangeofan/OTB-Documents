\chapter{Welcome}
\label{chapter:Welcome}

Welcome to the \emph{ORFEO ToolBox (OTB) Software Guide}.

This document presents the essential concepts used in OTB. It will
guide you through the road of learning and using OTB. The Doxygen
documentation for the OTB application programming interface is
available on line.

\section{Organisation}
\label{sec:Organisation}

This guide is divided in three parts, each of which is further divided
into chapters.

La premi\`{e}re partie pr\'{e}sente l'OTB de fa\c{c}on g\'{e}n\'{e}rale, comment proc\'{e}der \`{a} son installation et sa g\'{e}n\'{e}ration sur votre machine. 
Cette partie pr\'{e}sente donc les principes de bases d'architecture et de compilation sur un syt\`{e}me, et comment compiler une application en C++.

La deuxi\`{e}me partie pr\'{e}sente l'OTB d'un point de vue \emph{utilisateur}. Elle se pr\'{e}sente sous forme d'exemples illutr\'{e}s.

La troisi\`{e}me partie pr\'{e}sente l'OTB d'un point de vue \emph{d\'{e}veloppeur}.
Cette partie explique comment cr\'{e}er vos propres classes, comment faire \'{e}voluer le produit.

\section{Se familiariser avec l'OTB}
\label{sec:CommentApprendreOTB}

Il y \`{a} deux cat\'{e}gories d'utilisateurs de l'OTB :
\begin{itemize}
        \item Les d\'{e}veloppeurs de classes, qui cr\'{e}ent des classes C++. 
        \item Les utilisateurs des classes existantes pour d\'{e}velopper et g\'{e}n\'{e}rer leurs propres applications.
\end{itemize}

Nous vous recommendons d'\'{e}tudier les exemples. Vous pourrez ainsi compiler et ex\'{e}cuter les exemples distribu\'{e}s 
avec le code source diponible dans le r\'{e}pertoire \code{OTB/Examples}. Lire le fichier \code{OTB/Examples/README.txt} 
d\'{e}crivant les diff\'{e}rents exemples founis dans les sous-r\'{e}pertoires.

Il y \`{a} de plus, un ensemble de tests suffisamment document\'{e}s et disponibles dans le r\'{e}pertoire \code{OTB/Testing/Code} qui vous 
montrent comment peuvent \^{e}tre utilis\'{e}es les classes dans l'OTB.

\section{Organisation du logiciel}
\label{sec:OrganisationLogiciel}

En cours ....

\section{T\'{e}l\'{e}charger l'OTB}
\label{sec:DownloadOTB}
 
\index{Downloading}

L'OTB peut \^{e}tre t\'{e}l\'{e}charg\'{e}e \`{a} l'adresse Internet 
\begin{center} 
  \url{http://www.cnes.fr/HTML/Download.php}
\end{center}

\subsection{T\'{e}l\'{e}charger le 'Package'}
\label{sec:DownloadingReleases}

\index{OTB!downloading release}

Avant de d\'{e}marrer, vous pouvez consulter le document \code{GettingStarted.txt}\footnote{http://www.cnes.fr/HTML/GettingStarted.txt}. 
Il vous donne un aper\c{c}u sur la proc\'{e}dure \`{a} suivre pour le t\'{e}l\'{e}chargement et l'installation.

Choisir le fichier compress\'{e} \code{.zip} ou \code{.tgz}. Le premier est plut\^{o}t destin\'{e} pour l'environnement \emph{Microsoft Windows}, 
le second pour les environnements \emph{unix} ou \emph{linux}.

Extraire le contenu du fichier compress\'{e} (avec \emph{unzip} ou \emph{gunzip}) dans le r\'{e}pertoire \code{OTB} 
pr\'{e}alablement cr\'{e}\'{e} sur votre syt\`{e}me.
Vous \^{e}tes alors pr\^{e}t \`{a} proc\'{e}der \`{a} la configuration et l'installation du produit, 
d\'{e}crite au chapitre \ref{sec:CMakeforOTB} \`{a} la page \pageref{sec:CMakeforOTB}.

\subsection{T\'{e}l\'{e}charger depuis SVN}
\label{sec:DownloadingFromSVN}

\index{OTB!SVN repository}



Le code source de l'OTB est accessible via un serveur \href{http://subversion.tigris.org/}{Subversion SVN} (rempla\c{c}ant du c\'{e}l\`{e}bre CVS)
 

Pour acc\'{e}der \`{a} l'OTB via SVN (sous UNIX et Cygwin), utilisez la commande suivante :
\begin{verbatim}
svn ......
\end{verbatim}

Ceci permet de t\'{e}l\'{e}charger le r\'{e}pertoire \code{OTB} contenant l'ensemble du code source de la biblioth\`{e}que \emph{OTB}.

Vous pouvez ensuite configurer et installer l'OTB sur votre syst\`{e}me en suivant les instructions d\'{e}crites 
au chapitre \ref{sec:CMakeforOTB} \`{a} la page \pageref{sec:CMakeforOTB})


\subsection{Arborescence du produit}
\label{sec:DirectoryStructure}

L'OTB est organis\'{e} en trois principaux composants : la biblioth\`{e}que OTB (r\'{e}pertoire \code{OTB}), 
les applications de l'OTB (r\'{e}pertoire \code{OTB-Applications}) et la documentation associ\'{e}e (r\'{e}pertoire \code{OTB-Documents}).

Le code source ainsi que les exemples se trouvent dans le r\'{e}pertoire \code{OTB}; la documentation, 
le tutorial et les proc\'{e}dures d'installation se trouvent dans le r\'{e}pertoire \code{OTB-Documents} ; 
les applications plus complexes (de plus haut niveau) se trouvent dans le r\'{e}pertoire \code{OTB-Applications}.


L'\code{OTB} contient les r\'{e}pertoires suivants :
\begin{itemize}
        \item \code{OTB/Code} --- contient globalement l'ensemble du code source de la biblioth\`{e}que OTB
        \item \code{OTB/Documentation} --- contient la documentation de la biblioth\`{e}que OTB, produite par Doxygen
        \item \code{OTB/Examples} --- contient un ensemble d'exemples, utilis\'{e}s notamment pour pr\'{e}senter le concept de l'OTB et 
        \'{e}galement utilis\'{e} pour illustrer le guide de l'OTB
        \item \code{OTB/Testing} --- contient un certain nombre de programmes, utilis\'{e}s pour tester et valider la biblioth\`{e}que OTB. 
        Ces tests sont lanc\'{e}s via le moniteur de test de CMake \emph{ctest}.
        \item \code{OTB/Utils} --- contient les codes sources des biblioth\`{e}ques utilis\'{e}es par l'OTB
\end{itemize}

Le r\'{e}pertoire \code{OTB/Code} (le coeur du logiciel) est structur\'{e} de la fa\c{c}on suivante :
\begin{itemize}
        \item \code{OTB/Code/Common} --- d\'{e}finitions de macro, typedefs, et toutes autres classes "factoris\'{e}es" utilis\'{e}es par les autres composants de l'OTB.
        \item \code{OTB/Code/IO} --- classes d'entr\'{e}es/sorties pour l'acc\`{e}s aux images (encapsulation de GDAL et de CAI)
        \item \code{OTB/Code/ChangeDetection} --- les classes de d\'{e}tections de changements
        \item \code{OTB/Code/FiterExtraction} --- les classes contenant les primitives et descripteurs impl\'{e}ment\'{e}s
        \item \code{OTB/Code/Learning} --- les classes d'apprentissage supervis\'{e} (utilisant SVM)
        \item \code{OTB/Code/Visu} --- les classes de visualisation et des IHM graphiques (utilisant VTK et FLTK)
\end{itemize}

Le r\'{e}pertoire \code{OTB-Documents} contient les r\'{e}pertoires suivants :
\begin{itemize}
        \item \code{OTB-Documents/Latex} --- fichiers \LaTeX{} utilis\'{e}s pour la production de documents.
        \item \code{OTB-Documents/SoftwareGuide} --- fichiers \LaTeX{} utilis\'{e}s pour g\'{e}n\'{e}rer ce guide. 
        Les exemples illustr\'{e}s dans ce guide sont g\'{e}n\'{e}r\'{e}s \`{a} partir des codes sources, contenus 
        dans le r�pertoire \code{OTB/Examples}, traduits en \LaTeX{}
\end{itemize}

La documentation \code{OTB-Documents} est disponible via SVN en utilisant la commande :
\begin{verbatim}
svn ....
\end{verbatim}


Le r�pertoire \code{OTB-Applications} contient les r\'{e}pertoires suivants :
\begin{itemize}
        \item \code{OTB-Applications/Chgts} --- application int\'{e}ractive de d\'{e}tection de changements, utilisant l'\code{OTB}
        \item \code{OTB-Applications/Viewer} --- outil de visualisation d'images
        \item \code{OTB-Applications/Utils} --- contient divers utilitaires comme par exemple un g\'{e}n\'{e}rateur de quick-looks, 
        un outil d'extraction de r\'{e}gions d'int\'{e}r\^{e}t, un outil d'affichage des m\'{e}ta-donn\'{e}es des images et 
        un outil de pseudo-ortho-rectification automatique d'images
\end{itemize}

Pour acc\'{e}der aux applications de l'OTB via SVN (sous UNIX et Cygwin), utilisez la commande suivante :
\begin{verbatim}
svn ......
\end{verbatim}


\subsection{Documentation}
\label{sec:Documentation}

Associ\'{e}e \`{a} ce document, il existe deux autres documentations :
\begin{description}
        \item[La Documentation Doxygen .] La documentation Doxygen est une documentation essentielle pour d\'{e}velopper avec l'OTB. 
        Sous format HTML, elle d\'{e}crit en d\'{e}tail chaque classes et m\'{e}thodes impl\'{e}ment\'{e}es dans l'OTB. 
        Elle est illustr\'{e}e par des diagrammes de collaboration et des diagrammes d'h\'{e}ritage. 
        Cette documentation tr\`{e}s dynamique poss\`{e}de des hyper-liens sur les autres classes et sur le code source.
        Cette docmentation est disponible \`{a} l'adresse \url{http://smsc.cnes.fr/PLEIADES/Fr/A_prog_accomp.htm/}.
	\item[Les fichiers \emph{Header}.] Chaque classe de l'OTB est impl\'{e}ment\'{e}e dans un fichier .h et dans un fichier 
	.cxx/.txx (.txx pour les classes g�n�riques (\emph{template}) )
\end{description}

\subsection{Data}
\label{sec:Data}

Les images utilis\'{e}es dans ce guide proviennent de ............. 

