\chapter{Feature Extraction}
\section{Introduction}
What is feature extraction


\section{Interest Points}
\input{HarrisExample}
\section{Alignments}
\label{sec:Alignments}
\input{AlignmentsExample}
\section{Lines}
\label{sec:LineDetectors}

\subsection{Line Detection}
\label{sec:LineDetection}
\input{RatioLineDetectorExample}
\input{CorrelationLineDetectorExample}
\input{AssymmetricFusionOfLineDetectorExample}

\subsection{Segment Extraction}
\label{sec:SegmentExtraction}
\input{LocalHoughExample}
\input{ExtractSegmentsByStepsExample}
\input{ExtractSegmentsExample}

\section{Geometric Moments}

\subsection{Complex Moments}
\label{sec:ComplexMoments}
The complex geometric moments are defined as:
\begin {equation}
c_{pq} = \int\limits_{-\infty}^{+\infty}\int\limits_{-\infty}^{+\infty}(x + iy)^p(x- iy)^qf(x,y)dxdy,
\label{2.2}
\end{equation}
where $x$ and $y$ are the coordinates of the image $f(x,y)$, $i$ is the
imaginary unit and
$p+q$ is the order of $c_{pq}$. The geometric moments are
particularly useful in the case of scale changes.

\subsubsection{Complex Moments for Images}
\input{ComplexMomentImageExample}
\subsubsection{Complex Moments for Paths}
\input{ComplexMomentPathExample}

\subsection{Hu Moments}
\label{sec:HuMoments}
Using the algebraic moment theory, H. Ming-Kuel obtained a family of 7
invariants with respect to planar transformations called Hu invariants,
\cite{hu}. Those invariants can be seen as nonlinear combinations of
the complex moments. Hu invariants have
been very much used in object recognition during the last 30 years,
since they are invariant to rotation, scaling and translation. \cite{flusserinv} gives their expressions :

\begin{equation}
\begin{array}{cccc}
\phi_1 = c_{11};& \phi_2 = c_{20}c_{02};& \phi_3 = c_{30}c_{03};& \phi_4 = c_{21}c_{12};\\
\phi_5 = Re(c_{30}c_{12}^3);& \phi_6 = Re(c_{21}c_{12}^2);& \phi_7 = Im(c_{30}c_{12}^3).&\\
\end{array}
\end{equation}


\cite{dudani} have used these invariants for the recognition of
aircraft silhouettes. Flusser and Suk have used them for image
registration, \cite{flusser_2}.

\textbf{Examples}
\subsubsection{Hu Moments for Images}
\input{HuMomentImageExample}
\subsubsection{Hu Moments for Paths}
%\input{HuMomentPathExample}


\subsection{Flusser Moments}
\label{sec:FlusserMoments}
The Hu invariants have been modified and
improved by several authors. Flusser used these moments in order to
produce a new family of descriptors of order higher than 3,
\cite{flusserinv}. These descriptors are invariant to scale and
rotation. They have the following expressions:
\begin {equation}
\begin{array}{ccc}
\psi_1  = c_{11} = \phi_1; &  \psi_2  = c_{21}c_{12} = \phi_4; & \psi_3  = Re(c_{20}c_{12}^2) = \phi_6;\\
\psi_4  = Im(c_{20}c_{12}^2); & \psi_5  = Re(c_{30}c_{12}^3) = \phi_5;
& \psi_6  = Im(c_{30}c_{12}^3) = \phi_7.\\
\psi_7  = c_{22}; & \psi_8  = Re(c_{31}c_{12}^2); & \psi_9  = Im(c_{31}c_{12}~2);\\
\psi_{10} = Re(c_{40}c_{12}^4); & \psi_{11} = Im(c_{40}c_{12}^2). &\\

\end{array}
\end {equation}

\textbf{Examples}
\subsubsection{Flusser Moments for Images}
%\input{FlusserMomentImageExample}
\subsubsection{Flusser Moments for Paths}
%\input{FlusserMomentPathExample}


