\chapter{Online data}\label{sec:Online}

With almost every computer connected to the Internet, the amount of
online information is steadily growing. It is quite easy to retrieve
valuable information. OTB has a few experimental classes for this
purpose.

For these examples to work, you need to have OTB compiled with the
\texttt{OTB\_USE\_CURL} option to \texttt{ON} (and the curl library
installed somewhere).

Let's see what we can do.

\section{Name to Coordinates}
\label{sec:NamesToCoordinates}
\input{PlaceNameToLonLatExample.tex}


\section{Open Street Map}
\label{sec:OpenStreetMap}

The power of sharing which is a driving force in open source software such
as OTB can also be demonstrated for data collection. One good example is
Open Street Map (\url{http://www.openstreetmap.org/}).

In this project, hundreds of thousands of users upload GPS data and draw maps of their
surroundings. The coverage is impressive and this data is freely available.

It is even possible to get the vector data (not covered yet by OTB), but
here we will focus on retrieving some nice maps for any place in the world. The following
example describes the method. This part is pretty experimental and the code is
not as polished as the rest of the library. You've been warned!

\input{TileMapImageIOExample.tex}

