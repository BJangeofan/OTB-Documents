\section{OTB-Wrapping: bindings to Java language}
OTB-Wrapping is a project designed to allow classes from OTB 
to be wrapped for use with languages like Python, and Java and Tcl. 
The project is based on CMake. SWIG, CableSwig and finally OTB that must be already built.
 The OTB-wrapping project is an extension of the project WrapITK \url{http://code.google.com/p/wrapitk/}. 
It allows the Python, Java and Tcl users to have program written in their own language to use the classes and 
the capabilities of OTB.
Since OTB-Wrapping provides an interface to some C++ compiled code, execution times are very similar to C++ 
pure programs in most of cases.
Once the project compiled, one can use the wrapped classes to write programs in your favorite language.
The OTB Tutorials were translated in Java and can use the OTB capabilities to show how to use the 
Orfeo Toolbox easily. 

\subsection{Mangling}
You have probably notice that class names are quite barbaric. It is due to the mangling. 
With the Java versions less or equal to 1.6, no generic programming is allowed. That means that no templates\\
can be used. Then, to use a C++  class in Java, the class generated (called proxies) are templated with\\
the available types and for each type a Java proxy is generated. The mangled types are concatenated and added\\
in the end of the class. For instance, when one want to use the otb::Image with the pixel type float and the\\
dimension 2, the class generated will be otbImageF2. \\

The table ~\ref{tab:basictypetomangle} summarize the mangling used for the basic types. 

\begin{table}[!htbp]
\begin{center}
\begin{tabular}{|c|c|}
\hline
Type                  &  Mangle used  \\
\hline
float                 &  F         \\
double                &  D         \\
unsigned char         &  UC        \\
unsigned short        &  US        \\
unsigned long         &  UL        \\
bool                  &  B         \\

\hline 
\end{tabular}
\caption{Mangling of the basic types used.}\label{tab:basictypetomangle}
\end{center}
\end{table}

In OTB, there are some types used in several classes like otb::Image and otb::VectorImage,  or otb::VectorData.
Then those classes have their own mangling two. The table ~\ref{tab:typetomangle} shows some of the mangled classes. 

\begin{table}[!htbp]
\begin{center}
\begin{tabular}{|c|c|}
\hline
OTB Type                  &  Mangle used  \\
\hline
otb::Image                 &  I         \\
otb::VectorImage           &  VI        \\
otb::VectorData            &  VD        \\
\hline 
\end{tabular}
\caption{Mangling of the usually types used.}\label{tab:typetomangle}
\end{center}
\end{table}


The mangled name otbImageF2 means that we are dealing with the class otb::Image with 
template arguments float and 2. So, otbImageF2 is equivalent to \verb|otb::Image<float,2>|. Another example 
can be otbImageFileReaderIF2, knowing that I (~\ref{tab:typetomangle}) is the mangle for otb::Image, F is the mangle 
for the type float then IF2 is the mangle for \verb|otb::Image<float,2>|, and finally otbImageFileReaderIF2 is\\
the mangle for
 \verb|otb::ImageFileReader< otb::Image< float,2 > >|.


\subsection{How to Use OTB-Wrapping in Java}

\subsubsection{Import OTB classes}
The OTB-Wrapping project provides some jar files that corresponds to the modules implemented in the directory Libraries.\\
Each OTB-wrapping module is automatically wrapped to a separate Java package. For instance, the directory Libraries\\
contains the modules otbImages and otbLearning. The jars produced will be org.otb.otbimages and org.otb.otblearning.\\
Those jars can be used as usual in Java.

To write an example that uses OTB classes, you have to import those jars in your program by adding the directive 
\verb?import org.otb.otbio.*?. You can than access to all the classes in the module otbIO (cf section \ref{sec:tuto}).

\subsubsection{Compile Java programs}
The next step is to compile the Java program. Note that the Java compiler needs some options like 
\begin{itemize}
\item The class path  : specifies where to find user class files. \\
      The option is  -classpath .:path\_to\_the\_jar1:path\_to\_the\_jar2.
\end{itemize}

\subsubsection{Java programs execution}
Now we can execute the Java class compiled, it needs some options too.
\begin{itemize}
\item The class path  : specifies where to find user class files.\\
      The option is -classpath .:path\_to\_the\_jar1:path\_to\_the\_jar2...
\item The java.library.path : A system property that specifies where the libraries are.\\
      The libraries are named moduleJava.\{ dll|so|nylib \}.
\end{itemize}

\begin{landscape}
\subsection{Example}

Here is a simple example written in C++, and its translation in Java. 


\begin{table}[!htbp]
\begin{center}
\begin{tabular}{|p{8.35cm}|p{9.75cm}|}
\hline
Example in C++                  &  Example translated in Java  \\
\hline

\verb$#include "otbImage.h"$                &       \verb$import org.otb.otbimage.*;$                                 \\
\verb$#include "otbImageFileReader.h"$      &       \verb$import org.otb.otbio.*;$                                    \\
\verb$#include "otbImageFileWriter.h"$      &                                                                        \\
                                            &                                                                         \\
\verb$void otbReaderWriter(Int argc,char * argv[]){$ & \verb$public class otbReaderWriter{$        \\
                                                     &  \verb$public static void main(String argv[]){$  \\
      
 //typedefs :                              &     //The typedefs step are not necessary since no generic \\
                                           &     //programmation is allowed in Java.    \\

\verb$typedef otb::Image<float,2>  ImageT;$             &           \\
\verb$typedef otb::ImageFileReader<ImageT> ReaderType;$  &           \\
\verb$typedef otb::ImageFileWriter<ImageT> WriterType;$  &            \\

                       &                                           \\
// Instanciation :     &                      // Instanciation :   \\

\verb$ReaderType::Pointer reader = ReaderType::New();$    & \verb$otbImageReaderIF2 reader = new otbImageReaderIF2();$ \\ 
\verb$WriterType::Pointer writer = WriterType::New();$    & \verb$otbImageFileWriterIF2 writer=new otbImageFileWriterIF2();$  \\

 & \\
 // Set the filenames                           & // Set the filenames \\
\verb$reader->SetFileName(argv[1]);$            & \verb$reader.SetFileName(argv[0]);$ \\
\verb$writer->SetFileName(argv[2]);$            & \verb$writer.SetFileName(argv[1]);$  \\
&\\
 // Plug the pipeline                           & // Plug the pipeline  \\
 
\verb$writer->SetInput(reader->GetOutput());$   &  \verb$writer.SetInput(reader.GetOutput());$    \\ 
 &\\
 // Trigger the pipeline execution              &  // Trigger the pipeline  execution\\
\verb$writer->Update();$                        &  \verb$writer.Update();$  \\
 
\hline 
\end{tabular}
\caption{C++ code translation.}\label{tab:translatedexample}
\end{center}
\end{table}


\end{landscape}

\subsection{Use OTB-Wrapping}

\subsubsection{Download sources}
The source code for the project OTB-Wrapping is available here: \url{http://hg/orfeo-toolbox.org/OTB-Wrapping}. 

\subsubsection{Required Tools}
OTB-Wrapping required the release 3.2 of the Orfeo Toolbox or greater. This project required some tools and 
and libraries : 
\begin{itemize}
\item CMake (\verb!>2.6!) to configure the compilation system.
\item CableSwig (\verb!>0.1.0!) for parsing and automatic instanciation of C++ templates.\\
      The source code is available here: \url{http://FIXME}
\item SWIG  (\verb!>1.3.40!) for language binding
\item Target Language developement libraries (JRE for Java, Python 2.6 for Python).
\end{itemize}

\subsubsection{Compile OTB-Wrapping sources}
First, this project needs OTB, CableSwig and SWIG to be built. Once, this step done, create a 
directory where the project will be compiled (Don't compile in the sources directory).
The compilation can be done successfully following those steps :

\begin{itemize}
\item Run CMake (ccmake under Linux and macosX or cmake.exe under Windows) and specify the path to the sources.
\item Edit the paths for OTB, CableSwig and SWIG binary directories.
\item Turn the option WRAP\_ITK\_(JAVA | PYTHON | Tcl  )  to ON to choose your target language.
\item Choose the modules you want to wrap.
\item Configure the project and generate the solution.
\item Compile using make in the binary directory (if Linux or macosX), or open the solution (.sln)\\
      and generate the solution.
\end{itemize}

Once the compilation completed successfully, the directory lib will contains several jars which are Java archives 
and the libraries corresponding to the modules. Those are the useful to use OTB capabilities in Java language.

\subsubsection{Download binaries}
The binaries are available here :

\subsubsection{Use OTB-Wrapping with Eclipse}
OTB-Wrapping produces Java archives and libraries that can be used with Eclipse. You can then create your Java \\
project and use the OTB wrapped classes. 
Eclipse gives the possibility to add external jars. 
After creating a new project, 
\begin{itemize}
\item Run the project configuration (Run then Run Configurations) to add the externals OTB-Wrapping Jars.\\
      In the Classpath section, add external jars by searching the jars produced. (They are generally in directory lib).
\item Add the path to the OTB-Wrapping libraries to  the environment variable PATH. Go to the tab Environment \\
      (always in Run Configurations setup), \emph{Select} the environment variable PATH, and add the complete path \\
      to the location of the libraries.
\end{itemize}

You can know write Java classes, main and import the OTB wrapped classes like shown in section \ref{sec:tuto}.


\normalsize

\section{Java tutorials}\label{sec:tuto}
\lstsetjava
\subsection{Hello World}
\input{HelloWorldOTBJava.tex}

\subsection{Pipeline}
\input{PipelineJava.tex}

\subsection{Filtering pipeline}
\input{FilteringPipelineJava.tex}

\subsection{Smarter Filtering Pipeline}
\input{SmarterFilteringPipelineJava.tex}

\subsection{Scaling Pipeline}
\input{ScalingPipelineJava.tex}

\subsection{MultiSpectral}
\input{MultiSpectralJava.tex}

\subsection{OrthoFusion}
\input{OrthoFusionJava.tex}


\normalsize
\section{Python tutorials}\label{sec:tutoPython}
\lstsetpython
\subsection{Hello World}
\input{HelloWorldPython.tex}

\subsection{Pipeline}
\input{PipelinePython.tex}

\subsection{Filtering pipeline}
\input{FilteringPipelinePython.tex}

\subsection{Smarter Filtering Pipeline}
\input{SmarterFilteringPipelinePython.tex}

\subsection{Scaling Pipeline}
\input{ScalingPipelinePython.tex}

\subsection{MultiSpectral}
\input{MultiSpectralPython.tex}

% \subsection{OrthoFusion}
% \input{OrthoFusionPython.tex}



\section{Developper Guide}\label{sec:dev}

\subsection{Add a new Library : Creating a new CMakeList.txt file}
Each OTB-Wrapping sub-library (e.g. itkBase, or otbChangeDetection) 
lives in a sub-directory of the OTB-Wrapping project (within the Libraries directory)
 with a CMakeLists.txt file that describes how that library and its language support files 
(e.g. Java template definitions) is to be created. 
See SampleCMakeLists.txt in the Documentation directory for a description of each macro and option that
can appear in such a file. What follows is the usual set of commands that will appear:

\small
\verb$WRAP_LIBRARY("otbMySpatialObjectExtensions")$ \\
\verb$SET(WRAPPER_LIBRARY_DEPENDS itkSpatialObject itkBase)$ \\
\verb$SET(WRAPPER_LIBRARY_LINK_LIBRARIES ITKCommon )$ \\
\verb$AUTO_INCLUDE_MODULES()$ \\
\verb$END_WRAP_LIBRARY()$ \\
\normalsize

\begin{itemize}
\item \verb$WRAP_LIBRARY$  sets up the environment to wrap a set of classes into a 
      library with a given name. This macro is defined in TypedefMacros.cmake. 
\item \verb$WRAPPER_LIBRARY_DEPENDS$ stores the list of OTB-Wrapping libraries on which the current library\\
      depends (e.g. which libraries like otbImages or otbChangeDetection, are going to be used in the \\
      current library). 
\item \verb$WRAPPER_LIBRARY_LINK_LIBRARIES$ stores a set of other libraries to add at link time. 
These can be 3rd party libraries that you will use 
(be sure to properly set \verb$LINK_DIRECTORIES$ in this case), 
or more commonly, the OTB libraries that need to 
be linked in, like OTBCommon, OTBIO, etc.
\item \verb$AUTO_INCLUDE_MODULES()$  edits the list of the headers that have to 
      be included by scanning all the cmake files present in the modules. This macro is implemented in the 
      file TypedefMacros.cmake.

\item \verb$END_WRAP_LIBRARY()$ is needed to check several things. First, it checks if all the dependecies 
      of the current library are wrapped. Secondly, it calls the right macro relative to the language selected
      in order to creates rules to parse the CableSwig inputs and compile a wrapper library. 
\end{itemize}

\subsection{Add a new cmake file}
A \verb$wrap_XXX.cmake$ file defines a group of classes and/or template instantiations to be wrapped. 
Most of time, one such file is defined for each class wrapped, but this is not strictly necessary. 
Within such a file, directives are issued to wrap classes and particular template instances.
OTB-Wrapping define several macros and variables designed to:

\begin{itemize}
\item Make creation of wrappers easy. The syntax is simple enough to start quickly.
\item Make the choice of template arguments explicit. It should be easy 
      to understand the idea of the author of a wrapper by reading the file. 
\item Support mostly transparently the dimensions and types chosen by the user.
\end{itemize}

\subsubsection{A simple Example : otb::StreamingShrinkImageFilter}
This is the common case. Create a wrapper for a 
simple image filter like 
\verb$otb::StreamingShrinkImageFilter$. 

Here is what should be the file 
\verb$otbResampling/wrap_otbStreamingShrinkImageFilter.cmake$.

\small
\begin{verbatim}
WRAP_CLASS("otb::StreamingShrinkImageFilter" POINTER)
 WRAP_IMAGE_FILTER_USIGN_INT(2)
 WRAP_IMAGE_FILTER_SIGN_INT(2)
 WRAP_IMAGE_FILTER_REAL(2)
END_WRAP_CLASS()
\end{verbatim}
\normalsize

The file contains several macros useful to make the wrapping easy. It contains 
a \verb$WRAP_CLASS$ - \verb$END_WRAP_CLASS$ wich contains itself some 
\verb$WRAP_IMAGE_FILTER_*$ macros. 
\verb$WRAP_CLASS("otb::StreamingShrinkImageFilter" POINTER)$ begins the wrapping of the 
\verb$otb::StreamingShrinkImageFilter$ templated class.
The name of the class has to be fully qualified.  It can also contains two namespaces if it is 
a functor for example. The name will be \verb$otb::Functor::LHMI$.

The option \verb$POINTER$ indicates that the object of the class can be manipulated with a 
\verb$SmartPointer$, and the  \verb$SmartPointer$ specialization for the wrapped class has to be created.

Then several \verb$WRAP_IMAGE_FILTER_*$ macros are called. That are convenient macros to create
wrapper for classes which take only image as template argument. For each type supported  
a macro is defined. The argument 2 of the macros means that this class will be templated with 
two image types and they will be the same.


\subsection{Predefined Macros}

Those macro are defined and well documented in the file Typedef Macros.

\begin{itemize}
  
\item \verb$WRAP_CLASS("fully_qualified::ClassName" [POINTER|POINTER_WITH_SUPERCLASS])$
  causes a templated class to be wrapped. All namespaces must be 
  included in the fully qualified name but no template is given 
  at this point. All the wrap instructions are given between 
  \verb$WRAP_CLASS(...)$ and \verb$END_WRAP_CLASS()$using various \verb$WRAP$ directives. 
  This macro make an implicit call to the macro \verb$WRAP_INCLUDE(namespace::ClasseName.h)$ so the 
  header of the wrapped class don't have to be manually added. To disable this behavior, 
  set the variable \verb$WRAPPER_AUTO_INCLUDE_HEADERS$ to \verb$OFF$.
  
  The final optional parameter to \verb$WRAP_CLASS$ is \verb$POINTER$ or
  \verb$POINTER_WITH_SUPERCLASS$. If no options are passed, then the class is wrapped
  as-is. If \verb$POINTER$ is passed, then the class and the typedef'd 
  \verb$class::Pointer$ type is wrapped.

\item \verb$WRAP_INCLUDE("header.h")$. By default, 
  \verb$otbStreamingShrinkImageFilter.h$ is included
  when \verb$otb::StreamingShrinkImageFilter$ is wrapped, and this behavior is usually
  enough. If it not enough, this macro can be used to include some specific files.
  
\item \verb$WRAP_TEMPLATE("mangled_suffix" "template parameters")$. When is sued between 
  \verb$WRAP_CLASS$ and \verb$END_WRAP_CLASS$, this command causes a particular template 
  instantiation of the current class to be wrapped. The parameter \verb$mangled_suffix$ is a suffix to
  append to the class's name that uniquely identifies this particular template
  instantiation, and "template parameters" are whatever should go between the \verb$< >$
  template instantiation brackets. (Do not include the brackets.) if you are
  wrapping a filter.

\item \verb$WRAP_NON_TEMPLATE_CLASS("fully_qualified::ClassName"$\\\verb$[POINTER|POINTER_WITH_SUPERCLASS])$.
  Same as \verb$WRAP_CLASS$, but creates a wrapper
  for a non-templated class. No \verb$END_WRAP_CLASS()$ is necessary after this macro
  because there is no block of template instantiating commands to close.
  
\end{itemize}

There are other several well documented macros defined in the file TypedefMacros.cmake.


\subsubsection{OTB-Wrapping predefined variables}
OTB-Wrapping defines some pairs of variables for each basic type the developer may have
to manipulate: the C++ type, and its template parameter name. The name of the type
is stored in variable \verb$ITKM_*$, and the C++ type in the variable \verb$ITKT_*$.

For example, for \verb$unsigned char$, \verb$ITKM_UC$ and \verb$ITKT_UC$
are defined, with \verb|${ITKM_UC} = "UC"|  and \verb|${ITKT_UC} = "unsigned char"|.

The same thing is done for the defined classes. 
For instance, to retrieve the mangling used for  \verb|otb::Image<float, 2>|, 
it is stored in the variable  \verb|ITKM_IF2|.
Note that I corresponds to otb::Image following ~\ref{tab:typetomangle}.
The complete type will be stored in the variable \verb?ITKT_IF2?. 
These variables are used to make easier the mechanism of 
wrapping for the user (cf Section \ref{sec:dev})

 \subsubsection{OTB-Wrapping predefined lists}
The main task of the developer is to define which template parameters are valid for a given
templated class, and interesting for the user. He also have to take care about
instantiating some templated classes according to the options selected by the user.


With OTB-Wrapping, the developer don't have to declare that a class is instantiated with the template parameters
\verb$unsigned char$, \verb$unsigned short$, and \verb$unsigned long$, but rather declares that the
templated class can be instantiated with all the unsigned integer types chosen by the user.
To do that, OTB-wrapping provides some already defined list which are grouping the types chosen by
the user.

\begin{itemize}
  \item \verb$WRAP_ITK_DIMS$ contains all the dimensions selected by the user.

  \item \verb$WRAP_ITK_USIGN_INT$ contains all unsigned integer types selected by the user.

  \item \verb$WRAP_ITK_SIGN_INT$ contains all signed integer types selected by the user.

  \item \verb$WRAP_ITK_INT$ contains all signed and unsigned integral types
selected by the user.

  \item \verb$WRAP_ITK_REAL$ contains all the real types selected by the user.

  \item \verb$WRAP_ITK_SCALAR$ contains all the scalar types selected by the user.

  \item \verb$WRAP_ITK_RGB$ contains all the \verb$RGB$ types selected by the user.

  \item \verb$WRAP_ITK_VECTOR_REAL$ contains all the \verb$Vector$ types selected
by the user.

  \item \verb$WRAP_ITK_COV_VECTOR_REAL$ contains all the \verb$CovariantVector$ types selected
by the user.

  \item \verb$WRAP_ITK_VECTOR$ contains all the \verb$Vector$ and 
\verb$CovariantVector$ types selected by the user.

  \item \verb$WRAP_ITK_ALL_TYPES$ contains all the types selected by the user.

  \item \verb$SMALLER_THAN_D$ contains all the types "smaller" than \verb$double$
selected by the user. This variable is useful when a filter decrease the range
of pixel value, like \verb$BinaryThresholdImageFilter$.

  \item \verb$SMALLER_THAN_UL$ contains all the types "smaller" than \verb$unsigned long$
selected by the user.

  \item \verb$SMALLER_THAN_US$ contains all the types "smaller" than \verb$unsigned short$
selected by the user.

  \item \verb$SMALLER_THAN_SL$ contains all the types "smaller" than \verb$signed long$
selected by the user.

  \item \verb$SMALLER_THAN_SS$ contains all the types "smaller" than \verb$signed short$
selected by the user.

\end{itemize}

\subsubsection{Macro MANGLE\_NAME}
MANGLE\_NAME() is a macro used to make the mangling easier for the developper. 
This macro aims at formatting the mangled name using a separator. The separator can be user-defined by
setting the MANGLE\_SEPARATOR CMake value. This Macro takes as input a  list of types names.
At the end of the macro, the MANGLED\_NAME variable will contain the proper mangling. 
The macro was written to allow the user adding a separator to make the Java classes name more clear.
It is written in the file OTB-WrappingTypes.cmake. 

For instance, when one use the CMake macro  ADD\_TEMPLATE to add a template 
for the types used to wrap a class, a correct way to do so is :

\verb!ADD_TEMPLATE("${ITKM_IF2}${ITKM_IF2}" "${ITKT_IF2}, ${ITKM_IF2} ")!.

With the MANGLE\_NAME Macro, it will be in two steps : 
\begin{itemize}
\item\verb!MANGLE_NAME(${ITKM_IF2} ${ITKM_IF2} )!
\item\verb!ADD_TEMPLATE("${MANGLED_NAME}" "${ITKT_IF2}, \${ITKM_IF2}")!
\end{itemize}

\subsection{HTML JavaDoc generation}

All the classes wrapped in Java language (in Python too) contains the comments 
present in the C++ source code. The tags in the comments were translated to be 
specific to the tool Javadoc (@param, @return, @see...). 
In order to handle the Latex formulas present in the comments, the tool JavaDoc alone 
does not convert the Tex formulas to PNG. The taglet LatexLet allows to use Latex within 
JavaDoc comments. 

\subsubsection{Generate JavaDoc documentation}

First, download the taglet LatexLet 
here~:\\ \url{http://users.informatik.uni-halle.de/~grau/LaTeXlet/index.html}. 
Once the OTB-Wrapping project built, go to the directory Language/Java/Proxies/src/. 
All the java classes generated are there.

Then, to generate the javadoc HTML documentaion use the javadoc command with the following options : 

\begin{verbatim}
>javadoc -d DESTINATION 
         -taglet latexlet.InlineBlockLaTeXlet 
         -taglet latexlet.BlockLaTeXlet 
         -taglet latexlet.InlineLaTeXlet 
         -tagletpath PATH_TO_THE_LATEXLET_JAR   
          {BIN_DIR}/Languages/Java/Proxies/src/org/*/*/*.java
\end{verbatim}

The options mean :
\begin{itemize}
\item -d DESTINTAION : Destination directory for output files
\item -taglet *      : The fully qualified name of Taglet to register
\item -tagletpath PATH\_TO\_THE\_LATEXLET\_JAR : Complete path to the taglet jar
\item org/*/*/*.java : the sources files. 
\end{itemize}


\subsubsection{Generate JavaDoc while compilation}
The method above shows how to generate the HTML javadoc documentation with 
a command line after OTB-Wrapping was built. It is possible to include this step 
in the compilation process. 
For that, two cmake variables have to be edited in the project configuration step with cmake.
\begin{itemize}
\item  LATEXLET\_JAR : Path to the LatexLet taglet allowing to convert Tex formula to PNG. 
       Not setting this field will result in the absence of formulas in the generated javadoc
\item  WRAP\_ITK\_JAVADOC : by default on OFF. Put this variable to ON if to include the html \\
       javadoc generation in the compilation process.
\end{itemize}

The resulting documentation is located in BIN\_DIR/Languages/Doc/Javadoc.


